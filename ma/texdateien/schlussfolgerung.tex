% !TeX spellcheck = en_US
In conclusion, no solvent was found with high enough solubility to dissolve a lipid layer at cryogenic temperatures. Results show that solvents used for lipids are generally endothermic. Some solubility is reached at cryogenic temperature, but not enough to completely dissolve a lipid layer. Finding a solvent by chance is unlikely, as dissolving processes of lipids are majorly endothermic and therefore show low performance at cryogenic temperatures. 

The potential for lipids and detergents is not exhausted in this thesis. To engineer a sacrificial layer, other detergents could be used \cite{SigmaAldrich.2023}. An exothermic process is also not limited through the cyrogenic temperatures. Alternatively, to increase solubility, changing the pH value of lipid and solvent could increase solubility \cite{BruceA.AverillPatriciaEldredge.}.

PDMS is tuned at room temperature. Results have confirmed a connection between adhesion forces measured in papers and at tensile testing. Plasma activation has the opposite effect of 1:2 mixture ratio than on 50:1 mixture ratio. The surface gets more stable with increasing plasma activation instead of more brittle.

%At room temperature, the PDMS with different mixture ratios were tested. A mixture ratio of 1:2 has shown low adhesion forces as predicted. However, the transfer of results at room temperature to ice adhesion at cryogenic temperatures is only possible with restrictions. Plasma activation increases the adhesion forces potentially more on ice than the UV glue. At other mixture ratios, plasma treatment leads to brittle surfaces, which is not observed for 1:2 mixture ratio. Attempts of detaching an ice layer off this PDMS coated sample were not successful.

At cryogenic temperatures, a successful separation was made with 4:1 mixture ratio PDMS. A hole was observed suggesting a detachment. But the repeatability must be shown in future work, since limited tests are done with 4:1 mixture ratio.

All PDMS coated sample holders resulted in a continuous ice layers without cracks. PDMS of 1:2 with minimal plasma activation and 50:1 with high plasma activation resulted in ice layers which are not separable. No separation attempt was successful for both mixture ratio.

%Additionally, A PDMS mixture of 4:1 and 1:50 were investigated. With plasma treatment with $100\,\%$ power and \SI{10}{\minute}, Cracks form and the surface is brittle. However, The ice layer frozen on top of a PDMS coated sample is continuous. Ions produced in plasma treatment increase the adhesion force too much for the finger to detach. In future, stress needed to break the ice layer should be taken into account. Also to decrease the area of contact to the slide, a grid could be used to reverse the hydrophilic effect of plasma activation before freezing. The resulting small pieces of ice may be easier to detach than a clamped down continuous layer.

In contrast to PDMS coated sample holders, lipid coated sample holders resulted in cracks in the ice layer itself. In pulling tests with the finger, breaking and moving parts of the ice layer were possible in 1 out of 4 cases. To help detach the lipid layer, PDMS could be used. With increased brittleness and pre-broken ice layer, less stress is needed for detaching.

A combination of lipids and PDMS could be attempted. The brittleness of HFE and the cracks in the ice layer resulting from lipids could help decrease adhesion for a reliable detachment.

The finger with HFE is able to apply tensile stress between \SI{103.3}{\kilo\pascal} and \SI{585.4}{\kilo\pascal}. The stress applied is a couple magnitude higher than the stress calculated for detaching PDMS. As no detachment was made, additional factors increase tensile stress needed for detachment. The discrepancy could be result of energy lost deforming HFE at cryogenic temperatures, stability of a continuous ice layer, as well as the effect of plasma activation on some mixture ratios.

Different factors which increase and decrease the effectiveness of the finger: Too much HFE results in HFE ripping because of the low cohesion. Too little HFE is prone to not properly attaching to the Sample. lower temperature increase the viscosity and therefore stability of HFE. But at around \SI{-170}{\degreeCelsius}, cracks form in HFE and the strength is drastically reduced.

The direction of force could help at detaching an ice layer. However, the observed stability to shear load of HFE is considerably lower than tensile load.

Positioning is also important, especially keeping the correct gap between sample and finger in which the HFE is spread thin but not flows between window and sample. Also slow temperature changes are drastically reducing the likelihood of a good connection between sample and finger.

The force applied with the finger is not able to reliably break the ice layer on the sample. Pre-broken ice pieces are often picked up. A process is needed to completely loosen the ice layer without the "finger". For example, deactivating the plasma by putting a grid on the PDMS after Plasma activation could result in a loose, non-continuous but regular layer. Smaller pieces with less adhesion are easier to detach.

In general, the finger setup turned out to be not reliable. Even after analysis, forces applied with the finger vary too much for inducing strong forces to a sample. However, pre-broken pieces may be able to be picked up easier. Therefore, other methods of breaking loose the ice layer should be searched.

Another way to create a loose ice layer is to engineer the ice layer itself. Additives inside the ice layer could help in the breaking process. Freezing the ice in an emulsion could result in an in-continuous ice layer. 

