% !TeX spellcheck = en_US
In Experiments, no lipid solvent combination was found with high solubility at cryogenic temperatures. All tested solving processes are endothermic. In general, endothermic processes are inefficient in cold temperatures. To engineer a sacrificial layer, other detergents could be used \cite{SigmaAldrich.2023}. An exothermic process is also not limited through the cyrogenic temperatures.

Detaching an ice layer from a lipid layer is not reliable 

Different factors were examined on the effect of detaching of the finger. 

At room temperature, the PDMS with different mixture ratios were tested. A mixture ratio of 1:2 has shown low adhesion forces as predicted. However, the transfer of results at room temperature to ice adhesion at cryogenic temperatures is only possible with restrictions. Plasma activation increases the adhesion forces potentially more on ice than the UV glue. 

Generally, the "finger" is still not reliable after investigating the different factors and applying the insights. Additionally, issues with other assemblies like temperature controller and microscope reduce reliablity beforehand. 


\section{Outlook}

The potential for lipids and detergents is not exhausted in this thesis. lipids and detergents can be engineered for lower adhesion forces. Additionally, finding solvents by solely experiments is very unlikely. 

The force applied of the "finger" has proven to be not enough to break any surface on the sample. Pre broken pieces are sometimes picked up. A process is needed to completely loosen the ice layer without the "finger". For example, deactivating the plasma by putting a grid on the PDMS after Plasma activation could result in a loose, incontinuous but regular layer. Smaller pieces with less adhesion are easier to detach.

Another way to create a not continuous layer is with an imulsion. 

Engineer the ice layer itself

Other ways than the finger should be tried to loosen the ice layer
