% !TeX spellcheck = en_US
The finger with HFE is able to apply tensile stress between \SI{103.3}{\kilo\pascal} and \SI{585.4}{\kilo\pascal}. The stress applied is a couple magnitude higher than the stress calculated for detaching PDMS. The discrepancy could be result of energy lost deforming HFE at cryogenic temperatures and the stability of a continuous ice layer.

Different factors which increase and decrease the effectiveness of the finger: Too much HFE results in HFE ripping because of the low cohesion. Too little HFE is prone to not properly attaching to the Sample. lower temperature increase the viscosity and therefore stability of HFE. But at around \SI{-170}{\degreeCelsius}, cracks form in HFE and the strength is drastically reduced. The direction of force could help at detaching an ice layer. However, the observed stability to shear load of HFE is considerably lower than to tensile load. The ice structure is not negletable. A continuous ice layer has more adhesion to the sample than an already broken layer. Positioning is also important, expecially keeping the correct gap between sample and finger in which the HFE is spread thin but not flows between window and sample. Also slow temperature changes are drastically reducing the likelyhood of a good connection between sample and finger.

At room temperature, the PDMS with different mixture ratios were tested. A mixture ratio of 1:2 has shown low adhesion forces as predicted. However, the transfer of results at room temperature to ice adhesion at cryogenic temperatures is only possible with restrictions. Plasma activation increases the adhesion forces potentially more on ice than the UV glue. At other mixture ratios, plasma treatment leads to brittle surfaces, which is not observed for 1:2 mixture ratio. Attempts of detaching an ice layer off this PDMS coated sample were not successful.

Additionally, A PDMS mixture of 4:1 and 1:50 were investigated. With plasma treatment with $100\,\%$ power and \SI{10}{\minute}, Cracks form and the surface is brittle. However, The ice layer frozen on top of a PDMS coated sample is continuous. Ions produced in plasma treatment increase the adhesion force too much for the finger to detach. In future, stress needed to break the ice layer should be taken into account. Also to decrease the area of contact to the slide, a grid could be used to reverse the hydrophilic effect of plasma activation before freezing. The resulting small pieces of ice may be easier to detach than a clamped down continuous layer.

In contrast to PDMS coated slides, lipid coated slides resulted in cracks in the ice layer itself. In pulling tests with the finger, breaking and moving parts of the ice layer were possible in 1 out of 4 cases. To help detach the lipid layer, PDMS could be used. With increased brittleness and pre broken ice layer, less stress is needed for detaching.

In Experiments, no lipid solvent combination was found with high solubility at cryogenic temperatures. All tested solving processes are endothermic. In general, endothermic processes are inefficient in cold temperatures.

 The potential for lipids and detergents is not exhausted in this thesis. lipids and detergents can be engineered for lower adhesion forces. Additionally, finding solvents by solely experiments is very unlikely. 

To engineer a sacrificial layer, other detergents could be used \cite{SigmaAldrich.2023}. An exothermic process is also not limited through the cyrogenic temperatures. Alternatively, to increase solubility, changing the pH value of lipid and solvent could increase solubility \cite{BruceA.AverillPatriciaEldredge.}.

The force applied of the "finger" has proven to be not enough to break any surface on the sample. Pre broken pieces are sometimes picked up. A process is needed to completely loosen the ice layer without the "finger". For example, deactivating the plasma by putting a grid on the PDMS after Plasma activation could result in a loose, incontinuous but regular layer. Smaller pieces with less adhesion are easier to detach.

In general, the finger setup is not reliable. Even after analysis, forces applied with the finger vary too much for inducing strong forces to a sample. However, pre broken pieces may be able to be picked up easier. Therefore, other methods of breaking loose the ice layer should be searched.

Another way to create a loose ice layer is to engineer the ice layer itself. Additives inside the ice layer could help in the breaking process. Freezing the ice in an emulsion could result in an incontinuous ice layer. 

