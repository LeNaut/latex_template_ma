% !TeX spellcheck = en_US

To detach the ice layer from the sample holder, two main ideas are proposed: The first is using mechanical force to separate the ice layer. The second idea is designing a sacrificial layer which can be dissolved at cryogenic temperatures.

\section{Dissolving Phospholipids}

Phospholipids are made of two long nonpolar carbon chains and a polar head. A membrane is a bilayer of Phospholipids with the hydrophobic carbon chains oriented inwards and the hydrophilic head pointing outwards. They are also natural detergents, as they can bind to hydrophobic waste forming an emulsion, making them removable with polar liquids like water \cite{SriramaM.BhairiPh.D..2001}.

As Lipids are dissolvable at common solvents at room temperature, the idea is to use lipids as sacrificial layer at cryogenic temperatures. A solvent is used to dissolve the sacrificial lipid layer with a solvent. To make this possible, a solvent with high solubility of lipids must be found. As the dissolving process is temperature dependent, solvents at room temperature are not necessarily solvents at cryogenic temperatures. Experiments are conducted to find a solvent to dissolve lipids at cryogenic temperatures.

% as the surface of the sacrifical layer is only on the edge of the sample.!

Parylene is a hydrophobic polymer used as a coating to repel particles, including water and ice. Parylene is also biocompatible and used in medicine and biology \ref{comelec.}. Parylene is not usable without a second layer on top. Parylene hydrophobicity does not allow water to spread during plunge freezing. with plasma activation, the surface is now hyrdophilic, but ice adheres to the parylene too strong to mechanically detach. also parylene cannot be dissolved with a solvent as a sacrificial layer.

For this reason, lipids are used in combination with parylene (Fig. \ref{fig:sacrificial layer}). The hydrophobic chains of the lipids adhere to the parylene. The polar head allow water to spread evenly over the surface. Solving the lipids with a solvent will detach the ice layer from the slide. Parylene additionally prevents (re-) attachment through holes in the lipid layer. 

\begin{figure}[hbt!]
	\centering
	\input{../images/Zeichnung_Layer_Lipide_parylene.pdf_tex}
	\caption{Layers of a Sample. The Lipid layer is used as a sacrificial layer. To reach the layer with a solvent, the only contact surface is to the edge. To get a fast and reliable process, a solvent with high solubility is needed.}
	\label{fig:sacrificial layer}
\end{figure}

\section{Separating the ice layer mechanically}

The other method discussed in this master thesis is to mechanically lift a part or the whole ice layer from the sample holder it is frozen to. The sample is frozen with plunge-freezing. For later steps the sample is prepared in a bath. The "finger" assembly is used to attach and pull on the ice layer. Hydrofluorether (HFE) temperature dependent viscosity is used to connect and pull onto the ice layer with the finger.

The goal is to vary the engineered layer do reduce the adhesion between ice and sample holder. The parylene and lipid coated samples are tested regarding being separable with the finger. Additionally, PDMS is used for its general low adhesion to ice. with different mixture ratios and plasma treatment, PDMS is tuned to be hydrophile at freezing and keeping low adhesion when separating.
%TODO
PDMS is a polymer used in coatings for e.G. passive deicing. It is hydrophobic and has a low surface energy. Also it can be coat spinned into a thin layer to form the thin coat. Also it is widely available and tunable. To use PDMS in plunge freezing, the PDMS is plasma activated.

To create different PDMS mixtures, Dowsil Sylgard 184 Silicone elastomer is used\cite{DOW.}. The PDMS kit has two components. The Specified mixture ratio is 10 base coat to 1 curing agent in weight (10:1). In some applications, other mixture ratios are used and additives are added for tuning PDMS. In my research, I focused on tuning the mixture ratio of base coat to curing agent to reduce adhesion of the ice layer to the sample holder.  This is done in room temperature with tensile testing.

To find the correct mixture ratio of base coat and curing agent, tensile testing is done on PDMS coated cover glass pieces. A tensile testing machine is used. A stamp is glued on top of the PDMS layer with UV-glue. Then the stamp is pulled up while the cover glass piece is clamped to the bottom side. With the maximum force and area left behind of the UV-glue, the tensile stress is determined.

First, different mixture ratios and uncoated cover glass as a control is compared to another. Then, the effect of plasma curing is tested on PDMS at mixture ratios with low base coat portion. With these results, a mixture ratio with low adhesion is found to test at cryogenic temperatures. 

A small bath and later a big bath are placed below the finger. In the baths, samples are prepared and holds a harbor-shuttle system. The sample is fixed in the shuttle for transportation and to hold on to the sample holder when pulling with the finger. A warm nitrogen gas barrier keeps the finger tip ice free by displacing humid air.

To try out the separating with the finger in a repeatable manner, following process is used. The sample is plunge-frozen with fluoresceine water or taken out of the storage and placed on the work surface in the bath. The sample is fixed to a copper shuttle. The now prepared shuttle is transported quickly to the microscope for pre-imaging. After microscopy, the sample is placed into the bath unter the cooled finger. First, if not already done, cool down the finger to "unglue" mode. HFE is applied to the tip. The "finger" is lowered onto the sample while correcting the position with three stages until the HFE contacts and spreads over the sample. The temperature is reduced to "glue" temperature and waited until the sample and finger is cooled down. When the temperature is reached the finger is pulled up by turning the stage until detachment. Then the shuttle is transported to the microscope and the sample is analyzed.

When detachment is successful, the ice piece hanging on the "finger" is placed on another shuttle. This is done by lowering the finger onto the new shuttle and raising the Temperature to "unglue" mode.

The finger is also combined tensile testing machine. The finger is fixed to the top, while the bath is fixed to the bottom. The force at separation can be measured and quantified. The tensile strength of HFE can also be estimated. These results can be compared to data collected at pulling tests at room temperature and data in papers.

As the finger is used for this application for the first time, different variables are determined which could significantly influence successful detaching. With experiments, these variables are examined to improve the reliability of the finger. Following main variables are determined: The right volume of HFE on finger tip improves reliability. The temperature influences the strength of HFE. The right direction of force can make separation easier. Right positioning is important to induce forces efficient onto the right surfaces.