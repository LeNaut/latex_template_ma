% !TeX spellcheck = en_US


\section{Lipids}

In the previous chapter, the method of using a sacrificial layer to detach Ice was discussed. For this, lipids need to be solved at cryogenic temperatures. As not every lipid is solvable the same way in different solvents, a first test is conducted to obtain the potential solvents at room temperature. then the best solvents are also tested at cryogenic temperatures.

The solubility of lipids at room temperature in different solvents are tested. For this experiment the cover glasses are coated with lipids. Then a first reference image was taken. Then the cover glass is given into a small container with the potential solvent. After 15 minutes, the cover glass is removed and compared under the microscope with the reference picture. If streaks created from lipids are still as visible as before, the lipids are categorized as insoluble in this solvent. If the streaks partially dissapeared and/or are less visible, the lipids are categorized as partially soluble in this solvent. Last if the streaks completely disappear, the lipids are assinged as soluble in the solvent (Table \ref{table:LoeslichkeitRaumtemperatur}).


\begin{table}[hbt!]
	\centering
	\begin{tabular}{|l|c|c|}
		\hline
		potential solvent & solubility EGG-PC & solubility DOPC \\
		\hline
		\hline
		4-Methyl Pentene & soluble & N/A  \\ 
		\hline
		3-Methyl Pentene & slightly soluble & insoluble \\
		\hline
		1-Pentene & insoluble & insoluble \\
		\hline
		Isopentane & soluble & slightly soluble\\
		\hline
		1-Propanol & soluble & soluble\\
		\hline
		Pentane & soluble & insoluble\\
		\hline
		Ethanol & N/A & soluble\\
		\hline
	\end{tabular}
	\caption{result of solubility tests at room temperature. soluble indicates solvents which are able to visibly solve all lipids off a cover glass. slightly soluble indicates solutions which are able to solve lipids, but some stains are left: insoluble indicates no visible changes of tested lipid.}
	\label{table:LoeslichkeitRaumtemperatur}
\end{table}

This experiment shows that three different solvent exist for each EGG-PC as well as DOPC with high solubility (Table \ref{table:LoeslichkeitRaumtemperatur}). Following those results, solvents categorized with "soluble" are tested regarding solubility at temperatures of \SI{-140}{\degreeCelsius}. As not all solutions are liquid at \SI{-140}{\degreeCelsius} (Table \ref{table:SchmelztemperaturLösungsmittel}), they are tested at higher temperatures above their melting point, as mentioned in chapter \ref{chapter:meltingtemp}. In addition they are tested as mixtures with other solvents with a lower melting point, to lower its melting point. Additionally all lipids are tested in liquid ethane. Ethane was not tested at room temperature, as the boiling point is at \SI{-88.6}{\degreeCelsius} (ZITAT PUBCHEM ETHANE).

This experiment shows that no tested solvent was able to completely solve lipids at \SI{-140}{\degreeCelsius} and within \SI{15}{\minute} (Table \ref{table:Cryoloeslichkeit}). Also the smears of lipids did not only stay partially behind, but also new streaks appear on the glass slides. This means that some lipids redistributed on the glass slide.

Using solvents to destroy a sacrificial layer, a high solubility is a requirement. In this case, the sacrificial layer would be completely covered by the ice layer except the edges. So the solvents have only a small area to start solving the layer. To solve it completely, a strong solvent is needed to detach the ice layer from the slide. Additionally, as the ice layer needs to stay vitrified, the temperature cannot be raised over \SI{-140}{\degreeCelsius}. 

The solving process of lipids in solutions is probably endothermic. This means that heat is needed to solve lipids, so cold temperature heavily decrease solubility QUELLE DENNIS ODER SO. This effect was observed over the last experiments by all solvents to varying degree. It can be assumed that the majority of solvent lipids mixtures are endothermic which is very disadvantageous for finding a potential solvent lipid candidate. Strongly exothermic solvents could heat up the ice enough to create ice crystals, which would not be feasible. So only weakly exothermic solvents are feasible for this task. 

\begin{table}[hbt!]
	\begin{subtable}{\linewidth}
		\centering
		\begin{tabular}{|l|l|}
		\hline
		Solvent & Result \\
		\hline
		\hline
		Pentane & soluble at \SI{-125}{\degreeCelsius} \\
		\hline
		4-methyl pentene & insoluble \\
		\hline
		\makecell[l]{1:1 volume ratio\\ HFE to 1-Propanol} & \makecell[l]{did not mix,\\ slightly soluble}\\
		\hline
		Liquid ethane & insoluble\\
		\hline
		\end{tabular}
		\caption{EGG-PC}
		\label{table:EGG-PCCryoloeslichkeit}
	\end{subtable}
	\begin{subtable}{\linewidth}
		\centering
		\begin{tabular}{|l|l|}
		\hline
		Solvent & Result \\
		\hline
		\hline
		\makecell[l]{1:4 volume ratio\\ 1:2 molar ratio\\ Ethanol to Isopentane} & slightly soluble\\
		\hline
		\makecell[l]{1:2 volume ratio\\ 1:1 molar ratio\\ 1-Propanol to Isopentane} & insoluble \\
		\hline
		Isopentane & slightly soluble\\
		\hline
		1-Propanol & \makecell[l]{at \SI{-130}{\degreeCelsius}\\ slightly soluble}\\
		\hline
		Liquid ethane & insoluble \\
		\hline
		\end{tabular}
		\caption{DOPC}
		\label{table:DOPCCryoloeslichkeit}
	\end{subtable}
	\caption{ in \ref{table:EGG-PCCryoloeslichkeit} for EGG-PC, no sufficient solubility at -140°C was found. In \ref{table:EGG-PCCryoloeslichkeit}, DOPC was tested but also no proper solution was found.}
	\label{table:Cryoloeslichkeit}
\end{table}

As finding a good solvent lipid combinations seems very unlikely, a new method was tested. In the next section, the finger tool is used to try mechanically detach the ice layer.

\FloatBarrier
\section{Finger}
\label{Chapter:LipidPullingTests}

For this section, cover glass coated in Parylene are used as object slide. The slide is then dipped in solution containing lipids for a lipid coating. A ice layer with fluoriscine is frozen with either plunge-freezing or using a pincer and liquid nitrogen. Additionally, the "finger" is used as tool to try lifting off a piece of ice from the frozen layer on top of the lipids. In the next sections, different variables are examined and tested.

\subsection{Finding right dosage of HFE}

First obvious variable and potential issue source is the amount of HFE used as glue. High dosages of liquid hfe can spread underneath the frame holding the sample, leading to an inefficient force distribution. Also a big glue layer is a weak point between finger and sample, leading to a reduction of maximum force which can be applied. Too little glue will not connect the finger to the sample. Additionally, the dosaging of glue revealed to be a big challenge.

The HFE is dosaged with a pipette. The HFE is "taken up WORD" at room temperature, then the HFE is "released WORD" on the tip of the finger. In between, HFE is evaporating. Around $4\,\mu l$ is evaporating each time. Based on this knowledge, dosaging $4.10\,\mu l$, $4.30\,\mu l$ and $4.50\,\mu l$ is compared and a picture is made.

Results show that pipetting HFE is not reliable. The range spreads of too little to too much HFE even for those dosages. Not only differences in evaporation are playing a role. Correct placement on the tip is a major factor of glue dosaging. Still, a visual estimate for the correct glue dosage can be made by calculating the drop volume out of camera images.

\begin{figure}[hbt!]
	\centering
	\begin{subfigure}[]{0.45\textwidth}
		\includegraphics[width=6cm]{Temp_Picture_Lower_Limit}
		\caption{chosen example for lower limit}
	\end{subfigure}
	\begin{subfigure}[]{0.45\textwidth}
		\includegraphics[width=6cm]{Temp_Picture_Upper_Limit}
		\caption{chosen example upper limit}
	\end{subfigure}
	\caption{example of upper limit and lower limit for glue dosages. (BETTER PICTURES NEEDED)}
\end{figure}

To calculate the actual glue dosage, two exemplary Pictures of an Upper and lower limit of glue dosages is picked. Then the Volume is calculated with a formula for the volume of a spherical section. All needed components are calculated out of the estimated contact angle of the glue $\alpha \approx 45°$ and the tip diameter of $d = 1.68\,mm$, for the lower range a reduction of $d$ by a factor of $\frac{2}{3}$ is assumed as the drop is not covering the whole tip. The resulting volume range of the glue dosage is $ 0.11\,\mu l \gtrapprox V \gtrapprox 0.38\,\mu l $. Also the lower end of this range is desired, but the repetition range in correctly dosaging lower doses is lower.

\subsection{Temperature over applied force}

As the HFE gluing effect is temperature dependent, the temperature needs to be regulated precisely. To narrow in the temperature dependency of HFE, Different temperatures are tested. Also at some point, cracks form in HFE. Those cracks make HFE brittle so less stress can be put on HFE without breaking.

(SKIZZE?)

In application tests on lipid samples, the temperatures -150°C, -155°C, -160°C, -165°C and -170°C are compared. A needle is put in the HFE to subjecively observe the mechanical properties of HFE at certain temperatures.

At \SI{-150}{\degreeCelsius}, the HFE is still only lightly viscious and cannot hold up the needle. Reducing the Temperature to \SI{-155}{\degreeCelsius} results in more viscosity, but still not enough to hold up the needle. At \SI{-160}{\degreeCelsius} the HFE is viscious enough so that the Needle is hold up by the HFE. Also the Needle can be pulled out and the HFE is closing the gap. Also with enough force, the Needle can penetrate the HFE. Also at \SI{-165}{\degreeCelsius}, the the HFE gets more viscious, the Needle is harder to pull out or put in the HFE. also no cracks formed so far. Now at \SI{-170}{\degreeCelsius}, It hardens further, The HFE is still viscious, but with wiggling, the Needle can now be pulled out easier. Also if the Temperature is only a bit under \SI{-170}{\degreeCelsius}. Cracks form and the Needle is easily removable but penetration is impossible.

Heating the cracked HFE up results in cracks eventually disappearing. at \SI{-165}{\degreeCelsius}, first cracks disappear, but a lot remain, which is still lowering the mechanical stability of the HFE. With \SI{-160}{\degreeCelsius} still some cracks remain. Heating the HFE to \SI{-140}{\degreeCelsius} will result in cracks completely disappearing.

At \SI{-165}{\degreeCelsius}, maximum stress load can be applied. Higher temperatures have lower viscosity, which lowers the maximum stress before HFE breaks. At \SI{-170}{\degreeCelsius}, if the temperature not precisely regulated, eventually cracks will form, lowering the maximum stress. Also HFE is the weakest link between Finger and Shutte. So maximizing the HFE stability directly results in higher forces which can be applied onto the ice layer.

To test how applicable the temperatures are, pulling test are done as described in section \ref{Chapter:LipidPullingTests} except the temperatures of the finger is lowered to \SI{-165}{\degreeCelsius} and \SI{-170}{\degreeCelsius}. With extra attention to proper insulation and no leakage of the cold nitrogen gas, \SI{-165}{\degreeCelsius} is reachable and can be held over time. Still, it is not practical as leaks are sometimes spotted late. Then parts of the pipes need to be heated up, disconnected and then properly reconected and then cooled down again. \SI{-170}{\degreeCelsius} cannot be reached and held by the current finger. 

Therefore, the finger can be used at \SI{-165}{\degreeCelsius}. But for better reliablility, smaller improvements should be made to improve reliablility of the finger. Also if the Sample is cooled down too much accidentally, the setup should be heated up to \SI{-140}{\degreeCelsius} to iron out cracks which potentially formed at the reduced temperature.

\subsection{Tensile mode vs Shear mode}



\subsection{Detaching ice with finger of plunge freezed samples}

Next observed possible factor is the thickness of the ice layer. In the following, samples freezed with a plunge-freezer are compared to samples freezed with a pincer in liquid nitrogen. The results are categorized in 4 categories: Not successful pulls don't have visible changes of the flourescent ice layer, Partially successes are visible breaks or clear movement of ice parts on the ice layer, Successful liftoff is a missing piece and a visible piece on the finger, which could be used for future steps. In the results, there is no difference between Hand freezed and plunge-freezed samples regarding detachability. Therefore Ice thickness is not a factor which makes detaching ice easier. As both methods don't show success in detaching ice pieces, it could still be a relevant factor but not a thing which should make a certain solution magically work xD

\subsection{other observed error sources??}

Wrong positioning, forming of ice

\begin{table}
	\centering
	\begin{tabular}{|c|c|c|}
		\hline
		Category & Hand-freezed & Plunge-freezed \\
		\hline
		\hline
		count executed tries & 4 & 4\\
		\hline
		unsuccessful & 3 & 3\\
		\hline
		breaks/movement of ice & 1 & 1\\
		\hline
		piece lifted with finger & 0 & 0\\
		\hline		
	\end{tabular}
	\caption{Comparison of detachability between hand-freezed and plunge-freezed samples}
\end{table}

\section{PDMS}

To speed up the process of finding the right balance of PDMS Mixture ratio and plasma curing, experiments at room temperature done. For this, a pulling machine is used. First the sample is prepared. Then, the sample is clamped to the pulling machine. Then a plexiglass stamp is aligned on top of the sample. With UV glue, the stamp is glued to the PDMS layer on the sample. Before gluing, the Force and distance is set to zero on the pulling machine. After gluing, a couple minutes are waited so no further stress changes are ongoing from the gluing process. then the Machine is pulling on the sample with constant velocity. After detachment, the measurement is stopped. Afterwards, the stamp and the layer are analyzed under a microscope. The area is determined. With the maximum force and area, the maximum stress is calculated. Each experiment is repeated multiple times.

In between experiments, small variations are made: two different stamps are used, one has an area of \SI{2}{\milli\meter} x \SI{3}{\milli\meter} and another stamp is \SI{3}{\milli\meter} x \SI{3}{\milli\meter}. Since the area is measured afterwards, this should not have a significant effect on the results. Also, in the beginning, while waiting of the stress changes to subside, the pulling machine was inactive. with this, the Force before pulling will be higher as zero. Before pulling, the machine is set back to zero. After pulling, there is an offset between the neutral value and the value before because of zeroing, so the offset needs to be corrected. To avoid this, the pulling machine is set to zero force while waiting instead. Then no offset correction is needed. The offset correction and new method increased the accuracy between pull tests.

To verify the setup, samples coated with 4:1 and 1:2 curing agent to base coat weight ratio and uncoated coverglass used as slides are compared. The results show 2:1 mixture ratio with $87.3\pm19.9\,\si{\kilo\pascal}$ is easier to detach than 1:4 mixture ratio with $429.1\pm5.1\,\si{\kilo\pascal}$ (Fig. \ref{fig:vgl4:1zu1:2zuGlas}). Also glass without PDMS takes up a lot more tensile stress with $1161.5\pm111.5\,\si{\kilo\pascal}$. sometimes the machine is able to break the glass. Under the microscope, it is not visible if the PDMS layer itself was lifted from the glass or not.

\begin{figure}[hbt!]
	\centering
	\includegraphics[width=14cm]{plotVGLZugspannungPDMSMischungsverhaeltnisse}
	\caption{Comparison 4:1, 1:2 Base coat to curing Agent and glass without PDMS}
	\label{fig:vgl4:1zu1:2zuGlas}
\end{figure}

In literature, the ice adhesion on PDMS without plasma treatment is \SI{35}{\kilo\pascal}. for 2:1 and 5:1 the stress is between $60$ to \SI{80}{\kilo\pascal} \cite{IbanezIbanez.2022}. This is lower considerably lower than the experiment before. Therefore, one limitation is that the actual adhesion between ice and PDMS cannot be simulated by this experiment. Still, there is a correlation between the values and the experiment can give an insight of PDMS durability. In the end, if the separation happens between ice and pdms or pdms and glass are both good results. 

\begin{figure}[hbt!]
	\centering
	\includegraphics[width=14cm]{ForceOverTime}
	\caption{force over Time}
\end{figure}

In the next experiment, the effect of plasma curing is investigated. As the mixture ratio of 1:2 has the lowest adhesion, this experiment used this pdms mixture ratio. The same setup is used. Samples with a 2:1 weight ratio PDMS are additionally plasma treated before quickly clamping on the pulling machine. Even with low repetition rates, a clear tendency can be observed. With lower and stronger plasma treatment, the durable the PDMS Layer gets (Fig. \ref{fig:PlotPlasmaAktivierung}). Over the whole range, The needed stress sextubles. Because the repititon rate is low, the exact values should be treated cautiosly. Also the results are not applicable to other mixture ratios, as different behaviour in plasma activation was observed between 2:1 and 4:1 weight ratio. also no glass-like state was observed in 2:1 weight ratio mixture.


\begin{figure}[hbt!]
	\centering
	\includegraphics[width=14cm]{plot2_1PlasmaAktivierung}
	\caption{PDMS 2:1 Comparison between various Plasma curing strengths and durations.}
	\label{fig:PlotPlasmaAktivierung}
\end{figure}


TODO VGL
