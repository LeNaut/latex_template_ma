% !TeX spellcheck = en_US

%\section{Motivation}

Light microscopy is the simplest and most commonly used to magnify small objects and structures. There are many building varieties for different applications with different magnification ranges. In general, two building variation exist: Transmission light microscopy do let light pass trough the sample to create an image. Reflective light microscopy uses the reflecting light from the sample to image the sample. 

For example, some light microscopes have additional fluorescence filters to detect fluorescence signals. These filters allow only a specific wave length pass, emitted by the fluorescent molecules. As a light source, filtered white light and lasers can be used, since only a specific wavelength range excites the fluorescent molecules.

In light microscopy many new advancements are still made. The Abbe criterium is one limit for many light microscopes. However, multiple modified methods exist which are able to increase the resolution	 beyond this limit \cite{Heintzmann.2006}.

Electron microscopy is another imaging method being used: With electrons, resolutions down to the size of atoms are possible. The smallest structures are observable with this method. Even the distribution of certain elements is detectable. Also the creation of 3D images is possible. (MEHR VT LISTEN)

For Electron microscopy, water containing structures must be fixated. One way to archieve fixation of the specimen is to freeze the water to ice at cryogenic temperatures. With rapid cooling, vitrified ice which is amorphic and does not destroy structures can be reached. To hold the ice structure, all setups including microscopes to handle the specimen needs to be cooled below \SI{-120}{\degreeCelsius} to prevent the crystallization of the ice.

Conveniently, the process of sample preparation for cryo light- and electron microscopy is very similar. Still, a major difference in the preparation process exist. To reach transparency of the electron beam for cryo scanning electron microscopy (cryo-TEM), the sample is required to have a thickness of less then \SI{100}{\nano\meter} and containing only lighter elements. Additionally, a thin film in combination of a mesh is used as a slide. This grid does not deliver a regular background. Additionally, the sample needs to have a low absorbtion of light in the visible spectrum in case a laser is being used. If the contrast is too high, the sample is damaged trough the light absorption and resulting heat.

The sample holder used for light microscopy and electron microscopy is of very limited interoperablity. This grid does not deliver a regular background for reflective electron microscopy. Also 
a grid is not completely transparent for light for transmissive light microscopy. Since the grid absorbs visible light, limiting the use of lasers. The slides used in light microscopy are mostly too thick for electron microscopy. Sometimes heavier elements are used too. Additionally some form of metal grid must be added for electron microscopy. 

Nevertheless, combining Light- and electron microscopy on the same sample brings big advantages: The larger scale of light microscopy and the usa of multiple wavelengths combined with fluorescence and high resolution does make studying samples easier. In this master thesis, a method for using cryo light microscopy and cryo elecron microscopy on the same sample is examined. To make this possible, the sample must be transferred to different sample holders between cryo light microscopy and cryo electron microscopy without destroying the sample. To archieve this, a layer between  ice and sample holder is engineered to enable detachment and transfer to another sample holder (fig. \ref{fig:layersingeneral}).

\begin{figure}[hbt!]
	\centering
	\input{../images/Zeichnung_Layers_in_general.pdf_tex}
	\caption{Depiction of a sample. The specimen is frozen inside the ice layer. The sample holder and ice layer are connected by an additional engineered layer. In this master thesis, the additional layer is varied to allow separation of the ice including the specimen from the sample holder.}
	\label{fig:layersingeneral}
\end{figure}


\section{Task and requirements}

DAS MUSS ÜBERARBEITET WERDEN

In sample preparation for cryo microscopy, specimens such as cells are frozen inside a thin ice layer. the specimen can be stained with fluorescein before freezing for later observation with cryo light microscopy. Also a sample can be prepared to study with cryo-transmission electron microscopy (cryo-TEM). cryo-TEM allows us to see samples in an hydrated state. This is only possible in cryo-TEM, as liquid water would evaporate in vaccuum \cite{Danino.2012}.

For cryo light microscopy and cryo-TEM, plunge-freezing is used in sample preparation \cite{Danino.2012} \cite{Faoro.2018}. This can be done either manually or with a plunge-freezer, with comparable results. In practice, using a plunge-freezer gives more consistent results.

To successfully plunge-freeze a sample, following steps are taken: First, the slide is held by tweezers. Then a \SI{2}{\milli\liter} water drop containing the specimen is pipetted onto the hydrophilic slide. The water droplet is blotted with filter paper, creating a thin film of water which evaporates quickly. The tweezers holding the slide is shot in cold liquid under \SI{-140}{\degreeCelsius}, typically liquid ethane. The rapid temperature drop freezes the water into a thin vitrificated layer of ice. vitrificated ice has no crystal structure.

A vitrificated sample is needed as ice crystals damage the specimen. Vitrificated ice is created by cooling water abruptly to temperatures below \SI{-120}{\degreeCelsius} \cite{Wowk.2010}. To archive rapid cooling, the slide needs a high thermal conductivity to freeze the water quickly. For example, thin sapphire slides or a metal grid with a film are commonly used. Also the liquid which is used to freeze the sample should not possess the Leidenfrost effect, which prevents instant contact of the sample with the cold liquid. As liquid nitrogen is possessing the Leidenfrost effect, other coolants like liquid ethane are used. Additionall

Currently, no slide exist which is suitable for plunge-freezing, cryo light microscopy and cryo-TEM. in plunge-freezing, a hydrophile surface required to archieve a thin ice layer. Additionally the thermal conductivity of the slide needs to be high for the steep temperature drop needed to create vitificed ice. In cryo-TEM, the sample needs to be extremely thin and small. A metal mesh is used to hold the sample in place. Additionally, only light elements should be used in samples as heavier electrons will disturb the results in cryo-TEM. The mesh used in cryo-TEM is not suitable. The mesh is rough and inhomogeneous with big changes of reflectiveness for a reflected light microscope. for light transmitted through the sample, the light can only penetrate the sample through the holes. This will heavily influence the image quality in cryo light microscopy.
 
To perform cryo-light microscopy and cryo-TEM in high quality, a sample transfer from one slide to another slide is proposed. The slide change must be performed at -140°C to maintain the vitrificated state of the sample. Additionally as the first slide used for plunge freezing needs to be hydrophilic, which increases ice adhesion in general. First, I investigated lipids for positive characteristics for usage as a sacrificial layer. Second, different layers are tested for low ice adhesion. Also different parameters on the mechanical setup are tested.

\section{State-of-the-art}

Ice removal is needed in multiple commercial applications. Most strategies are only applicated in temperatures down to \SI{-30}{\degreeCelsius}. Since the ice layer in the demonstrated application needs to stay vitrificated at under \SI{-140}{\degreeCelsius} or even lower, only few anti-frosting methods are feasible. For example the active anti-frosting strategy of heating the ice is not possible, as the specimen must stay contained within the ice layer.

There are four passive anti-frosting strategies: 

- Inhibition of ice nucleation is archieved by using surface inherent properties to prevent ice crystals from forming. 
- Retardation of frosting removes water to prevent icing on the surface with water repellent properties such as the lotus effect.
- Mitigation of frost accumulation prevents already formed ice droplets from further accumulating and forming an ice layer. 
- Last a reduction of ice adhesion on the surface prevents ice droplets to stay on the surface. Other forces like wind or gravity removes ice droplets and keeps the surface free of ice \cite{Yang.2021}. 

Out of the four passive anti-frosting strategies, only one is applicable at cryogenic temperatures: (Welche?) Die hier nicht, weil ? Inhibition of ice nucleation, retardation of frosting and mitigation of accumulation only inhibit the freezing process itself. Still, the reduction of ice adhesion can be used to decrease the force needed to detach the ice layer mechanically. This can be done by using a hydrophobic surface with general low adhesion. In industrial application, one commonly used coating is Polydimethylsiloxane (PDMS).

PDMS is a polymer which is widely used in different applications like in fabrication of microchannels, chip manufacturing, aerospace industry and medical tools. PDMS properties are for example its hydrophobility, biocompatability and electric insulating capabilities. PDMS is cost effective and allows rapid prototyping, molding and thin coatings. \cite{Wolf.2018}. Additionally, PDMS is modifiable with additives.

One example in which the low ice adhesion potential of PDMS is illustrated is the passive deicing of Aircrafts in flight. As ice can influence the air flow around the wing an the body, which induces turbulence and reduces lift. Ice protection is therefore critical for a save and stable flying. In \cite{Liu.2018}, PDMS is tuned for optimal characteristics in flight. To test the surfaces, flight conditions of \SI{0.5}{\bar} and \SI{-12}{\degreeCelsius} are simulated. Fluorinated PDMS with and without silica nanoparticles are compared to aluminum, showing better resistance against ice growth. The different coatings are also examined regarding contact angle of water and surface roughness. Also the stability of the surface is relevant as ice formation and impacts can also wear down the coating itself. 

There are multiple factors which influence the adhesion of ice on a surface. Hydrophility and the contact angle of water on a surface is for example closely related to ice adhesion. This suggest that plunge freezed samples, which requires a certain hyrdophility, are hard to detach. The adhesion is also dependent on temperature. Experiments performed close to freezing point may not be transferrable to the same setup at \SI{-140}{\degreeCelsius}. Additionally, the bulk material and the elasticity of ice itself changes over temperature \cite{Makkonen.2012}.
