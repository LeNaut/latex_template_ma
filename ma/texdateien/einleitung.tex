% !TeX spellcheck = en_US

Light microscopy (LM) is the simplest and most commonly used to magnify small objects and structures. There are many building varieties for different applications with different magnification ranges. In general, two building variation exist: Transmission LM do let light pass trough the sample to create an image. Reflective LM uses the reflecting light from the sample for imaging. 

For example, some light microscopes have additional fluorescence filters to detect fluorescence signals. These filters allow only the specific wave length emitted by a fluorescent molecule to pass. As a light source, filtered white light and lasers can be used, since only a specific wavelength range excites fluorescent molecules.

In LM many new advancements are still made. The Abbe criterium is the major limit for many light microscopes. However, multiple modified LM methods exist which are able to increase the resolution beyond this limit \cite{Heintzmann.2006}.

Electron microscopy (EM) is another imaging method being used: With electrons, resolutions down to the size of atoms are possible. The smallest structures in sub nanometer scale are observable with this method. Even the distribution of certain elements is detectable. Also the creation of 3D images are possible.

In EM, water containing structures must be fixated. One way to archive fixation of the specimen is to freeze the water to ice at cryogenic temperatures. With rapid cooling below \SI{-120}{\degreeCelsius} within tenths of milliseconds, vitrified ice which is amorphic and does not destroy structures by forming ice crystals is reached  \cite{Wowk.2010}. This requires low thermal conductivity which is only archievable with small and well thermal conducting sample holders. To hold the ice structure, all setups including microscopes to handle the specimen needs to be cooled below \SI{-120}{\degreeCelsius} to prevent subsequent crystallization of the ice.

%TODO Wiederholung, 2 Additionally
To reach transparency of the electron beam for cryo scanning electron microscopy (cryo-TEM), the sample is required to have a maximum thickness of \SI{100}{\nano\meter} and containing only lighter elements. Additionally, a thin film in combination of a metal grid is used as a sample holder \cite{Danino.2012}. This grid does not deliver a regular background. Additionally, the sample needs to have low absorption of light in the visible spectrum in case a laser being used. If the contrast is too high, the sample is damaged trough light absorption and resulting heat.

%TODO Hier absatz mit light microscopy and sapphire disk?
For cryo-LM, the thin ice layer is frozen onto a thin sapphire disk. also here the thickness is limited by thermal conductivity to reach the vitrified state. The surface needs to be transparent in visible light or with low contrast \cite{Faoro.2018}.

Conveniently, the process of sample preparation for cryo light- and electron microscopy is very similar. Still, tshe sample holder used for light microscopy and electron microscopy are very limited interoperable. This grid does not deliver a regular background for reflective electron microscopy. Also a grid is not completely transparent for light for transmissive light microscopy. also the grid absorbs visible light, limiting the use of lasers. The slides used in light microscopy are mostly too thick for electron microscopy. Sometimes heavier elements are used too. Additionally some form of metal grid must be added for electron microscopy. 

%TODO?
%Currently, no slide exist which is suitable for plunge-freezing, cryo light microscopy and cryo-TEM. in plunge-freezing, a hydrophile surface required to archieve a thin ice layer. Additionally the thermal conductivity of the slide needs to be high for the steep temperature drop needed to create vitificed ice. In cryo-TEM, the sample needs to be extremely thin and small. A metal mesh is used to hold the sample in place. Additionally, only light elements should be used in samples as heavier electrons will disturb the results in cryo-TEM. The mesh used in cryo-TEM is not suitable. The mesh is rough and inhomogeneous with big changes of reflectiveness for a reflected light microscope. for light transmitted through the sample, the light can only penetrate the sample through the holes. This will heavily influence the image quality in cryo light microscopy.

Nevertheless, combining Light- and electron microscopy on the same sample brings big advantages: The larger scale of light microscopy and the use of multiple wavelengths combined with fluorescence and high resolution on the same sample does make studying samples easier. In this master thesis, a method for using cryo light microscopy and cryo elecron microscopy on the same sample is examined. To make this possible, the sample must be transferred to different sample holders between cryo light microscopy and cryo electron microscopy without destroying the sample. To archieve this, a layer between ice and sample holder is engineered to enable detachment and transfer to another sample holder (fig. \ref{fig:layersingeneral}).

\begin{figure}[hbt!]
	\centering
	\input{../images/Zeichnung_Layers_in_general.pdf_tex}
	\caption{Depiction of a sample. The specimen is frozen inside the ice layer. The sample holder and ice layer are connected by an additional engineered layer. In this master thesis, the additional layer is varied to allow separation of the ice including the specimen from the sample holder.}
	\label{fig:layersingeneral}
\end{figure}


\section{Task and requirements}

%TODO IWIE KURZ

There are several requirements for this engineered layer: The layer must be thin to keep thermal conductivity high at freezing. The layer also must be hydrophile and even to  enable freezing a regular thin ice layer on top. The engineered layer should not disturb light microscopy or electron microscopy. 

%TODO Vielleicht absatz plunge-freezing?

%In electron microscopy, a mesh combined with an extremely thin carbon film as sample holder is used. As the carbon film is hydrophobic, plasma activation is typically used to increase wettability. For cryo-TEM, the ice layer including every additional layer should not be thicker than \SI{100}{\nano\meter}. Only "light" atoms are used to not disturb the image in cryo-TEM \cite{Danino.2012}. 

%In sample preparation for cryo microscopy, specimens such as cells are frozen inside a thin ice layer. the specimen can be stained with fluorescein before freezing for later observation with cryo light microscopy. Also a sample can be prepared to study with cryo-transmission electron microscopy (cryo-TEM). cryo-TEM allows us to see samples in an hydrated state. This is only possible in cryo-TEM, as liquid water would evaporate in vaccuum \cite{Danino.2012}.

The order of which kind of microscopy is performed first has an influence on the design of the layer. As the thickness and the design of the sample holder is less restrictive at light microscopy, designing the layer for this environment is also easier. Therefore, light microscopy is performed first in the final process. Then the ice layer is transferred to a grid. After this, cryo-EM is performed.

There are several challenges to perform a sample holder change: First, high wettability is needed when plunge-freezing. Wettability highly increases adhesion of ice on the rest of the sample. Second, the engineered layer is thin and covered by ice and sample holder. Therefore access to the layer is limited by solvents or other liquids. Third, the ice layer must stay in a vitrified state in the whole process. Last, many characteristics are temperature dependent. Results of experiments done at room temperature or close to freezing point are not easily transferable at \SI{-140}{\degreeCelsius}. For example, even mechanical stability and adhesion forces of all layers varies over temperature. Therefore everything needs to be validated at cryogenic temperatures \cite{Makkonen.2012}.

%TODO WAS WOLLTE ICH HIER SAGEN, DRÜber lesen, schauen dass sich das nicht zu arg doppelt
To fulfill all the requirements, following ideas are pursued. A sacrificial layer between sample holder and ice could be dissolved. The ice layer would float separately in the solution and could then be transferred to another sample holder. The second idea is to mechanically separate a piece or the hole ice layer from the sample holder. The layer must be designed in a way to reduce adhesion and posses high wettability at the same time. Also the forces applied must be strong enough for separation.

%First, I investigated lipids for positive characteristics for usage as a sacrificial layer. Second, different layers are tested for low ice adhesion. Also different parameters on the mechanical setup are tested.

\section{State-of-the-art}
\label{section:Stateoftheart}

Ice removal is needed in multiple commercial applications. Most strategies are only applicable in temperatures down to \SI{-30}{\degreeCelsius}. Since the ice layer in the demonstrated application needs to stay vitrificated at under \SI{-140}{\degreeCelsius} or even lower, only few anti-frosting methods are feasible. For example the active anti-frosting strategy of heating the ice is not possible, as the specimen must stay contained within the ice layer.

There are four passive anti-frosing strategies: Inhibition of ice nucleation is achieved by using surface inherent properties to prevent ice crystals from forming. Retardation of frosting removes water to prevent icing on the surface with water repellent properties such as the lotus effect. Mitigation of frost accumulation prevents already formed ice droplets from further accumulating and forming an ice layer. Last a reduction of ice adhesion on the surface prevents ice droplets to stay on the surface. Other forces like wind or gravity removes ice droplets and keeps the surface ice free \cite{Yang.2021}. 

Out of the four passive anti-frosting strategies, only one is applicable at cryogenic temperatures. The reduction of ice adhesion can be used to decrease the force needed to separate the ice layer mechanically. This can be done by using a hydrophobic surface with general low adhesion or with surface structuring. In contrast, Inhibition of ice nucleation, retardation of frosting and mitigation of accumulation only inhibit the freezing process itself and are therefore not applicable.

In industrial application, one commonly used coating to reduce adhesion of ice is Polydimethylsiloxane (PDMS). PDMS is a polymer which is widely used in different applications like in fabrication of microchannels, chip manufacturing, aerospace industry and medical tools. PDMS properties are for example its hydrophobility, biocompatability and electric insulating capabilities. Also PDMS is cost effective and allows rapid prototyping, molding and thin coatings \cite{Wolf.2018}. Additionally, PDMS is modifiable with additives.

One example in which the low ice adhesion potential of PDMS is illustrated is the passive deicing of Aircrafts in flight. As ice can influence the air flow around the wing an the body, which induces turbulence and reduces lift. Ice protection is therefore critical for a save and stable flying. In \cite{Liu.2018}, PDMS is tuned for optimal characteristics in flight. To test the surfaces, flight conditions of \SI{0.5}{\bar} and \SI{-12}{\degreeCelsius} are simulated. Fluorinated PDMS with and without silica nanoparticles are compared to aluminum, showing better resistance against ice growth. The different coatings are also examined regarding contact angle of water and surface roughness. Also the stability of the surface is relevant as ice formation and impacts can also wear down the coating itself. 

To create PDMS surfaces in labs Dowsil Sylgard 184 Silicone elastomer is widely used\cite{DOW.}. The PDMS kit includes a curing agent and base coat component. The specified mixture ratio is 10 base coat to 1 curing agent in weight (10:1). In some applications, other mixture ratios are used and additives are added for tuning PDMS.

For example, adhesion forces of ice on PDMS varies with different mixture ratios. In \cite{IbanezIbanez.2022}, PDMS is investigated in 3:1 to 1:50 weight ratio. The mixture is given into a mold and afterwards fully cured. An ice block is frozen on top of the PDMS. Using a pulling machine the maximum tensile and shear forces for detaching ice from PDMS are determined. Results show that mixture ratios 10:1 to 1:3 have significant lower adhesion forces on ice. At for example at 2:1, the shear mode is just under \SI{20}{\kilo\pascal} and the tensile mode around \SI{30}{\kilo\pascal}. 

Plasma curing is commonly used as PDMS treatment. Plasma treatment used for increasing wettability and adhesion. Plasma treatment is changing the chemistry of the polymer chains on the surface. Charged Oxygen Ions are deposited on the surface. these Ions make the surface temporarily hydrophile and increasing water adhesion. The Ions change the structure of the PDMS is permanently by oxidation. In some cases, cracks form as the surface oxidizes to a silica like form.

In \cite{Owen.1994}, the influence of different gasses on PDMS plasma treatment are examined. Oxygen, Nitrogen, Argon and Helium are compared on the effect on wettability, adhesion and cracking. Thin PDMS sheets are used with unknown composition. They found similar results between gasses. All gasses produced a thin and brittle surface with cracks and high wettability. Based on these results, used gas is not a significant factor for plasma activation. Therefore using only air (mainly a mixture of nitrogen and oxygen) is sufficient to determine the effect of plasma treatment.

In \cite{Ohishi.2017}, the influence of plasma treatment on Mixture ratios of 50:1 to 100:1 is described. A thin layer of those PDMS mixtures is put onto a preformed PDMS piece with lower mixture ratio. The preformed piece is used to apply shear stress to the surface by stretching the lower PDMS form piece. Therefore only tensile force are examined. It shows no significant difference between described mixture ratios. It also shows that higher plasma treatment leads to more brittle surfaces, reducing the force needed to break the PDMS Layer by $90\,\%$.

%&The effect of Plasma activation between mixture ratios is mostly unknown. At mixture ratios above 50:1 base coat to curing agent, no significant differences between plasma activation is observed \cite{Ohishi.2017}. 


