% !TeX spellcheck = en_US

%\section{Motivation}

Cells are the main building block of live. With light microscopes cells are made visible and can be studied by biologist. Many methods use fluorescence to study certain cell structures or processes in detail. For example live and dead cells are detectable with fluorescence. Still, not everything is visible even with the best light microscopes.

Light microscopy has a major limitation. The resolution of most microscopes is limited through the Abbe criterium. Many efforts are made to push and work around the limit \cite{Heintzmann.2006}. Still, other microscopy methods are needed to see the smallest of scale. Elecron microscopy is able to resolute up to rows of Atom scale. In Biology, electron microscopy is used to see viruses and other small structures which would be otherwise invisible. 

In electron microscopy, no live cells can be observed. The sample is put in a vacuum while electron microscopy. Gas would disturb the electron beam, making electron microscopy impossible. In vacuum at room temperature, Water evaporates and destroys the cell. One option is to replace the water with another substrate. The commonly used substrate is toxic, making it hard to work with. Another way is to freeze the water to ice at cryogenic temperatures. The microscope needs to be cooled to prevent the sublimation of the ice.

Light microscopy is also possible with frozen samples. Conveniently, the process of sample preparation is very similar. Still, a major difference in the preparation process exist. For electron microscopy, a thin film in combination of a mesh is used as a grid. This grid does not deliver a regular background. Also a grid is not completely transparent for light. Still, the combination would bring a big advantage. The larger scale of light microscopy and the usage of multiple wavelengths with fluorescence and high resolution on the same sample will make studying samples easier.

In this master thesis, a method for using cryo light microscopy and cryo elecron microscopy on the same sample is searched. To make this possible, the sample must be transferred between cryo light microscopy and cryo electron microscopy without damaging the sample. In this thesis, multiple methods to change the objective slides are tested and evaluated.

\section{Task and requirements}

In sample preparation, The Specimen such as cells are frozen inside a thin ice layer. the specimen can be stained with fluorescein before freezing for later observation with cryo light microscopy. Also a sample can be prepared to study with cryo-transmission electron microscopy (cryo-TEM). cryo-TEM allows us to see samples in an hydrated state. This is only possible in cryo-TEM, as liquid water would evaporate in vaccuum \cite{Danino.2012}.

for cryo light microscopy and cryo-TEM, plunge-freezing is used  in sample preparation \cite{Danino.2012} \cite{Faoro.2018}. This can be done either by hand or with a plunge-freezer. Generally the same result can be produced by hand as with a plunge freezer. In practice a plunge-freezer gives more consistent results. 

To successfully plunge-freeze a sample, following steps are taken: First, the slide is held by tweezers. Then a \SI{2}{\milli\liter} water drop containing the specimen is pipetted onto the hydrophilic slide. The water droplet is blotted with filter paper, creating a thin film of water which evaporates quickly. The tweezers holding the slide is shot in cold liquid under \SI{-140}{\degreeCelsius}, typically liquid ethane. The rapid temperature drop freezes the water into a thin vitrificated layer of ice. vitrificated ice has no crystal structure.

A vitrificated sample is needed as ice crystals damage the specimen. Vitrificated ice is created by cooling water abruptly to temperatures below \SI{-120}{\degreeCelsius} \cite{Wowk.2010}. To archive rapid cooling, the slide needs a high thermal conductivity to freeze the water quickly. Also the liquid which is used to freeze the sample should not possess the Leidenfrost effect, which prevents instant contact of the sample with the cold liquid. As liquid nitrogen is possessing the Leidenfrost effect, other coolants like liquid ethane are used.

But currently, no slide exist which is suitable for plunge-freezing, cryo light microscopy and cryo-TEM. in plunge-freezing, a hydrophile surface required to archieve a thin ice layer. Additionally the thermal conductivity of the slide needs to be high for the steep temperature drop needed to create vitificed ice. In cryo-TEM, the sample needs to be extremely thin and small. A metal mesh is used to hold the sample in place. Additionally, only light elements should be used in samples as heavier electrons will disturb the results in cryo-TEM. For light microscopy, a transparent slide is not always required. Still, a smooth surface is needed. The mesh used in cryo-TEM is not suitable. The mesh is rough and has holes, which influence the reflectiveness and light which is behind the sample(???). This will heavily influence the image quality in cryo light microscopy.
 
To perform cryo-light microscopy and cryo-TEM in high quality, a sample transfer from one slide to another slide is proposed. The slide change must be performed at -140°C to maintain the vitrificated state of the sample. Additionally as the first slide used for plunge freezing needs to be hydrophilic, which increases ice adhesion in general. First, I investigated lipids for positive characteristics for usage as a sacrificial layer. Second, different layers are tested for low ice adhesion. Also different parameters on the mechanical setup are tested.

\section{State-of-the-art}

Ice removal is needed in multiple commercial fields. But most strategies are applicated in temperatures down to \SI{-30}{\degreeCelsius}. Since the ice layer in my application needs to stay vitrificated at under \SI{-120}{\degreeCelsius} or lower, only few solutions are transferrable. For example the active anti-frosting strategy of heating the ice is not possible. The ice layer needs to stay intact and cannot be melted.

%\subsection{General strategies for ice removal}

There are four passive anti-frosing strategies: Inhibition of ice nucleation is archieved by using surface inherent properties to prevent ice crystals from forming. Retardation of frosting removes water to prevent icing on the surface with water repellent properties such as the lotus effect. Mitigation of frost accumulation prevents already formed ice droplets from further accumulating and forming an ice layer. Last a reduction of ice adhesion on the surface prevents ice droplets to stay on the surface. Other forces like wind or gravity removes ice droplets and keeps the surface ice free \cite{Yang.2021}. 

Out of the four passive anti-frosting strategies, only one is applicable at cryogenic temperatures. Inhibition of ice nucleation, retardation of frosting and mitigation of accumulation only inhibit the freezing process itself. Still, the reduction of ice adhesion can be used to decrease the force needed to detach the ice layer mechanically. One material commonly used is Polydimethylsiloxane (PDMS).

%\subsection{PDMS in application}

PDMS is a polymer which is widely used in different applications like in fabrication of microchannels, chip manufacturing, aerospace industry and medical tools. 
It is used because of its properties like hydrophobility, biocompatability, electric insulating. Also PDMS is cost effective and allows rapid prototyping, molding and thin coatings. \cite{Wolf.2018}. 

One application is the passive deicing of Aircrafts in flight. As ice can influence the air flow around the wing an the body, which induces turbulence and reduces lift. Ice protection is therefore critical for a save and stable flying. In \cite{Liu.2018}, PDMS is tuned for optimal characteristics in flight. To test the surfaces, flight conditions of \SI{0.5}{\bar} and \SI{-12}{\degreeCelsius} are simulated. Fluorinated PDMS with and without silica nanoparticles are compared to aluminum, showing better resistance against ice growth. The different coatings are also examined regarding contact angle of water and surface roughness. Also the stability of the surface is relevant as ice formation and impacts can also wear down the coating itself.  
