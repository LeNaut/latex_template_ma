% !TeX spellcheck = en_US

\section{Motivation}
Motivation:Biologische Proben sowohl unterm Licht- als auch Elektronenmikroskob anschauen

\section{Task and requirements}

In Application, samples (for example cells) are frozen inside a thin ice layer. The sample can be stained with fluorescence and observed with cryo light microscopy. Also a sample can be prepared to study with cryo-transmission electron microscopy (cryo-TEM) . This allows to see samples in an hydrated state. This is only possible in cryo-TEM, as liquid water would evaporate in vaccuum \cite{Danino.2012}.

For sample preparation in cryo light microscopy and cryo-TEM, plunge-freezing is used \cite{Danino.2012} \cite{Faoro.2018}. This can be done either by hand or with a plunge-freezer, but both methods use the same steps. First, the slide is held by tweezers. then a 2 to 4 ml water drop including the sample is pipetted on the hydrophilic slide. Therefore the droplet spreads over the slide. Then the water droplet is blotted with filter paper, creating a thin film of water which is evaporating quickly. Then the slide is shot in cold liquid under -140°C, for example liquid ethane. A temperature drop of over 100°C freezes the water into a thin ice layer. With this procedure, vitificated ice is formed without a crystal structure.

A vitrificated sample is needed as ice crystals damage the samples and can disturb cryo light microscopy. Vitrificated ice is created by freezing water abruptly into temperatures under -120°C \cite{Wowk.2010}. To create a vitrificated sample, the slide needs a high thermal conductivity to freeze the water quickly. Also the liquid which is used to freeze the sample should not possess the Leidenfrost effect. The vapors which are created with the leidenfrost effect will prevent a rapid temperature drop. As liquid nitrogen has the leidenfrost effect, other liquids like liquid ethane are used.

The motivation for this master thesis is to find a way to use cryo light microscopy and cryo-TEM on the same sample. But currently, no slide is found which has all requirements to be used in plunge-freezing, cryo light microscopy and cryo-TEM. in plunge-freezing, a hydrophile surface is needed to archieve a thin ice layer. Additionally the thermal conductivity of the slide needs to be high for the steep temperature drop needed to create vitificed ice. For light microscopy, a transparent slide is not always required. But a good thermal conductivity is advantageous as less energy is needed to keep the sample cool (WAS FÜR VORRAUSSETZUNGEN GIBT ES DA?). In cryo-TEM, the sample needs to be extremely thin and small. Additionally, only light elements should be used as heavier electrons are disturbing the image in cryo-TEM.

To perform cryo-light microscopy and cryo-TEM, a sample transfer from one slide to another slide is proposed. The slide change must be performed at -140°C to maintain the vitrificated state of the sample. Additionally as the first slide used for plunge freezing is hydrophilic, lifting the sample is not simply possible without designing a new layer. First, I investigated lipids for potential positive characteristics for a sacrificial layer or detaching it mechanically. then I am trying PDMS and use different mixture ratios and plasma curing to make mechanical removal easier.


\section{State-of-the-art}

\subsection{Methods for getting rid of ice xD}

There are four passive anti-frosing strategies: Inhibition of ice nucleation is archieved by using surface inherent properties and heating to prevent ice crystals to form. Retardation of frosting removes water over time to prevent icing on the surface with water repellent properties such as the lotus effect. Mitigation of frost accumulation prevents already formed ice droplets to further accumulate and forming an ice layer. Last a reduction of ice adhesion so ice needs less force to detach of a surface even until ice droplets are not able to attach to a surface, detaching themself with the force of gravity \cite{Yang.2021}. 

\subsection{PDMS application (in z.b. der Flugindustrie)}

PDMS is a polymer which is widely used in different application. It is used because of its properties like hydrophobility, biocompatability, electric insulating. Also PDMS has low costs and allows rapid prototyping \cite{Wolf.2018}. 

One application is the passive deicing of Aircrafts in flight. The Ice can influence the air flow around the wing, creating turbulence and reduce lift. Ice protection is therefore critical for a save flight \cite{Liu.2018}. In this paper, PDMS is tuned for optimal characteristics in flight. To test the surfaces, flight conditions of \SI{0.5}{\bar} and \SI{-12}{\degreeCelsius} are simulated. Fluorinated PDMS and Fluorinated PDMS is compared to aluminum, showing better resistance against ice growth. The different coatings are also examined regarding contact angle and surface roughness. Also the stability of the surface is relevant as ice can also wear down the coating itself.  



