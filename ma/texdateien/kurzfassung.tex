% !TeX spellcheck = en_US
Light microscopy and Electron microscopy are both widely used method to examine samples. Both microscopy method are able to observe different resolution and features. Results of both methods complement each other in observation.

As light microscopy and electron microscopy are fundamentally different, the preparation of samples is different. 

Fixation of specimen with vitrificated ice is used in cryo light microscopy and electron microscopy. Still, the sample holder cannot be used interchangeably between microscopy methods. Therefore, a new process is proposed to change the sample holder between microscopy steps.

In this master thesis, methods for separating ice layer including specimen from the sample holders are evaluated. A layer between sample holder and ice layer is introduced. 

The first idea is to solve a lipid layer between ice and sample holder for separation. A sacrificial layer containing lipids is designed. The sample holder is coated in parylene to decrease adhesion of ice on the sample holder. Solvents are tested regarding solubility at room- and cryogenic temperature. At room temperature, the lipids DOPC and EGG-PC are dissolvable in multiple common solvent. At cryogenic temperatures, the same solvents are not able to solve lipids at a satisfying rate. As most solvents of lipids are endothermic, %TODO MORE POSITIVITY

The second idea is to mechanically detach the ice layer. To archieve this, a layer between sample and ice layer is used to reduce adhesion. A lipid and parylene layer as well as PDMS layer are examined. 

To be able to mechanically lift off the ice layer without destroing the specimen, all assemblies are cooled to \SI{-140}{\degreeCelsius}. A lifting assembly named "finger" uses the temperature dependent viscosity of Hydrofluorether to attach and detach to the ice layer. Baths filled with liquid nitrogen are used for sample preparation. A modified inverted fluorescence microscope is used to confirm successful detachment.

PDMS is tuned to reduce adhesion. The effect of plasma treatment is researched by tensile testing at room temperature on 1:2 base coat to curing agent ratio. The results show an increase of tensile strength of PDMS of up to 6 fold of untreated PDMS. On the other hand, this is not the case for mixture ratio with higher base coat content. The PDMS layer gets brittle and reduces adhesion in that way.

At cryogenic temperatures, sample holders coated with plasma treated PDMS and parylene/lipids are compared to each other. A successful detachment was archieved with a PDMS mixture ratio of 4:1. The assemblies used can be improved on the test done. 

In future, %TODO Iwie positiv?