% !TeX spellcheck = en_US
Light microscopy and Electron microscopy are both widely used tools to study samples. Light microscopy makes detailed observation possible with visible light. Additionally, fluorescence can be used to highlight certain areas of interest within a sample. In electron microscopy, structures to the sub-nanometer is made visible by using a focused electron beam. The distribution of different atoms is observable. The strengths of both microscopy methods complement each other in analysis. 
%TODO zweitletzter satz

As both microscopy methods are fundamentally different, sample preparation fulfill different conditions to enable good visibility in according microscopy method. Still in cryo microscopy, both use similar preparation steps. In both a vitrified ice  is created to fixate the specimen inside the ice layer on the sample holder. The sample holder are designed to enable the applied microscopy method. Therefore, examining the sample in cryo-microscopy is limited by sample holder design.

In this master thesis, methods to switch sample holders at cryogenic temperatures are investigated. A additional layer is designed between sample holder and ice layer to enable detachment. Two different approaches of layer design are explored: Using a sacrificial layer to separate the ice containing the specimen and reducing adhesion of the ice layer to the sample holder to make mechanical separation possible.

To design a sacrificial layers, lipids are investigated. In theory, solvents are dissolving the sacrificial lipid layer at cryogenic temperatures. The sample holder is coated in parylene to decrease adhesion of ice and avoid (re-) attachment after dissolving. Solvents are tested regarding solubility at room temperature and around \SI{-140}{\degreeCelsius}. Results show that dissolving lipids is generally an endothermic process. Dissolving a DOPC and EGG-PC lipid layer at cryogenic temperatures is possible but not applicable for dissolving a sacrifical layer. 

To be able to mechanically lift off the ice layer without destroying the specimen, multiple assemblies are used. A lifting assembly named "finger" uses the temperature dependent viscosity of Hydrofluorether to attach and detach to the ice layer. Baths filled with liquid nitrogen are used for cooling and sample preparation. An inverted fluorescence microscope with a cryo stage is used to confirm successful detachment.

At room temperature, PDMS is tuned to reduce adhesion. The effect of plasma treatment is researched by tensile testing on 1:2 base coat to curing agent ratio. The results show an increase of tensile strength of PDMS of up to 6 fold of untreated PDMS. On the other hand, this is not the case for mixture ratio with higher base coat content. The PDMS layer gets brittle and reduces adhesion in that way.

At cryogenic temperatures, sample holders coated with plasma treated PDMS and parylene/lipids are compared to each other. Breaking the ice layer is succesful in $\frac{1}{4}$ th of cases with parylene/lipid coated sample holders. A successful detachment was archieved once with a sample holder coated with PDMS mixture ratio of 4:1.

Future work could focus on increasing solubility of lipids and other detergents by tuning pH-value. Also other methods for loosening the ice layer should be investigated. 

