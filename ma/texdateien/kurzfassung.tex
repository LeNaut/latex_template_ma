% !TeX spellcheck = en_US
Light microscopy and Electron microscopy are both widely used method to examine samples. Light microscopy makes observation possible over the visible light spectrum. In combination, fluorescence can be used to stain certain parts of a sample. In electron microscopy, structures to the smallest of scale is made visible by an electron beam. The distribution of different atoms is also observable. This complements both microscopy methods with each other.

As both microscopy methods are fundamentally different, sample preparation has different goals to enable each microscopy method. Most similar is the fixation of specimen with vitrificated ice is used in cryo light microscopy and cryo electron microscopy. In preparation, different sample holders are used. Generally, it is possible to combine cryo-light microscopy and cryo electron microscopy on the same sample with strong limitation. This results as sample holders optimized for one microscopy method are not designed for the other microscopy method.

In this master thesis, methods to switch sample holders are investigated. For this, two methods for separating ice layer including specimen from the sample holders are evaluated. A layer between sample holder and ice layer is designed. The layer is used to enable detachment of the ice layer to switch the sample slide between microscopy methods. 

The first idea is to dissolve a lipid layer between ice and sample holder for separation. A sacrificial layer containing lipids is designed. The sample holder is coated in parylene to decrease adhesion of ice on the sample holder. Solvents are tested regarding solubility at room- and cryogenic temperature. Results show that dissolving lipids is generally an endothermic process. This shows that dissolving the lipids DOPC and EGG-PC at cryogenic temperatures is too slow to dissolve a sacrificial layer. 

%TODO
The second idea is to mechanically detach the ice layer. To archieve this, a layer between sample and ice layer is used to reduce adhesion. A lipid and parylene layer as well as PDMS layer are examined. 

To be able to mechanically lift off the ice layer without destroying the specimen, all assemblies are cooled to \SI{-140}{\degreeCelsius}. A lifting assembly named "finger" uses the temperature dependent viscosity of Hydrofluorether to attach and detach to the ice layer. Baths filled with liquid nitrogen are used for sample preparation. A modified inverted fluorescence microscope is used to confirm successful detachment.

PDMS is tuned to reduce adhesion. The effect of plasma treatment is researched by tensile testing at room temperature on 1:2 base coat to curing agent ratio. The results show an increase of tensile strength of PDMS of up to 6 fold of untreated PDMS. On the other hand, this is not the case for mixture ratio with higher base coat content. The PDMS layer gets brittle and reduces adhesion in that way.

At cryogenic temperatures, sample holders coated with plasma treated PDMS and parylene/lipids are compared to each other. A successful detachment was archieved with a PDMS mixture ratio of 4:1. In future, assemblies should be improved and new ways of loosening the ice layer should be researched.

