Das ist die Methode.

\section{Voraussetzungen (Was kommt nicht in frage?)}

\section{Lipide}

\subsection{Vorbereitung Lipide Deckgläser}

\subsection{Lösbarkeit Lipide}

\subsubsection{Raumtemperatur}

\subsubsection{Cryotemperaturen}

\section{Finger}

\subsection{Plunge-freezing???}

\subsection{klebermengentest}
Am Anfang wurde die Klebermenge mithilfe einer Pinzette dosiert. Hierfür wurde das HFE zuallererst in ein Bad gegeben, welches auf -140??°C gekühlt wurde. Bei diesen Temperaturen verdampft das HFE nicht mehr. außerdem wird das HFE Dickflüssig. Mit einer Pinzette wird dann nun die benötigte Menge auf die Fingerspitze aufgetragen. Jedoch kann die genaue Menge auf dem Finger nur qualitativ abgeschätzt werden.

Um die optimale Klebermenge quantitativ zu bestimmen, wurde das HFE mit Hilfe einer Pipette (MARKE) Pipettiert. 

\subsection{temperaturtest}

\subsection{Test ablösen Lipide}

\section{PDMS}

warum PDMS so intensiv.

\subsection{Vorbereitung PDMS Proben}

Für eine 4:1 Beschichtung eines Glases wird folgendes
Verfahren angewendet: zunächst wird das PDMS im
verhältnis (hier 4 base coat zu 1 curing agent gewichtsverhältnis)
abgewogen und gut zusammengemischt.
zum entfernen der Luftblasen wird das PDMS für
30 minuten in eine Vakuumglocke gegeben. Die
zu beschichtenen Glasobjektträger werden so lange mit
ethanol/1-Propanol gereinigt. Das PDMS wird
anschließend mit eine Coatspinner auf die Glasobjekt-
träger dünn aufgetragen. Anschließend wird das PDMS
für 30 min bei 80°C ausgehärtet.
Für ein Mischungsverhältnis war 1:2 wird eine deutlich
Längere Aushärtzeit von mindestens 24h. benötigt, sonst härtet das
Das PDMS nicht komplett aus, welches unter anderem bei der Plasmaaktivierung in unbeabsichtigte effekte resultiert. 

\subsection{Versuch einfluss der Plasmabehandlung auf die Ablösekraft}

\subsubsection{Versuchsaufbau?}

\subsection{PDMS Ablöseversuche bei cryotemperaturen}

1:2 als auch 4:1

