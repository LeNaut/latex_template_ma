% !TeX spellcheck = en_US

Das ist die Methode.

\section{Task and requirements}

In Application, samples (for example cells) are frozen inside an thin ice layer. Te sample can be stained with fluorescence and observed with cryo light microscopy. Also a sample can be prepared to study with cryo-transmission electron microscopy (cryo-TEM) . This allows to see samples in an hydrated state. This is only possible in cryo-TEM, as liquid water would evaporate in vaccuum \cite{Danino.2012}.

For sample preparation in cryo light microscopy and cryo-TEM, plunge-freezing is used \cite{Danino.2012} \cite{Faoro.2018}. This can be done either by hand or with a plunge-freezer. In both ways, the slide is first held by tweezers. then a 2 to 4 ml water drop including the sample is pippetted on the hydrophilic slide. The droplet spreads over the whole slide. Then the water droplet is blotted with a filter paper, creating a thin film of water which is evaporating fast. Then the slide with the sample is put inside of a cold liquid like liquid ethane. The cold liquid should not have the Leidenfrost effect, which makes a needed temperature drop of over 100°C in milliseconds possible. With this procedure, vitificated ice is formed without a crystal structure. A vitrificated sample is needed as ice crystals damage the samples and can disturb cryo light microscopy.

The main motivation for this master thesis is to find a way to use cryo light microscopy and cryo-TEM on the same sample. But currently, no slide is found which has all requirements to be used in plunge-freezing, cryo light microscopy and cryo-TEM. in plunge-freezing, a hydrophile surface is needed to archieve a thin ice layer. Additionally the thermal conductivity of the slide needs to be high for the steep temperature drop needed to create vitificed ice. For light microscopy, a transparent slide is not always required. But a goot thermal conductivity is advantageous as less energy is needed to keep the sample cool (WAS FÜR VORRAUSSETZUNGEN GIBT ES DA?). In cryo-TEM, the sample needs to be extremely thin and small. Additionally, only light elements should be used as heavier electrons are disturbing the image in cryo-TEM. As those requirements are very specific, a slide change could work around those issue.

Still, the slide change must be performed at -140°C to maintain the vitrificated state of the sample. Additionally as the first slide used for plunge freezing is hydrophilic, lifting the sample is not simply possible without designing a new layer. First, I investigated lipids for potential positive characteristics for a sacrificial layer or detaching it mechanically. then I am trying PDMS and use different mixture ratios and plasma curing to make mechanical removal easier.

\section{Phospholipids}

Phospholipids are the building block of membranes in nature. They own two long hydrophobic chains and a polar head. the membrane is formed by a bilayer with the hydrophobic parts showing inwards and the hydrophilic head pointing outwards. They are also natural detergents, as they can bind to hydrophobic DIRT and forming an emulsion, making them removable with polar liquids like water \cite{SriramaM.BhairiPh.D..}.

Phospholipids are generally solvable in Alcohols(???). To apply Phospholipids, the solution is given on the surface. When the solvent dries out, the lipids are bining to the surface creating a layer. this layer can be solved again with the same solvents. If the ice layer is held by lipids, they can be used as a sacrificial layer, being solved at cryogenic temperatures. But to solve this layer, a high solubility at cryogenic temperatures must be given as the surface of the sacrifical layer is only on the edge of the sample.

\subsection{Parylene}

One idea of balancing Hydrophilic and Hydrophobic characteristics is to use Parylene. Parylene is superhydrophobic (SOURCE??? MAYBE NOT), which helps ice not to adhere to the surface. But used alone, an ice layer could not be frozen on top with plung-freesing or by hand as a water drop would not spread on the surface. 

For this reason, lipids are used in combination with parylene. The lipids are holding the Ice layer onto the parylene. With a solvent, the lipids can be solved and the parylene will prevent the ice layer to hold on the slide, detaching the layer. Or mechanical pulling on the ice is easier, as parylene is preventing not perfectly covered pieces from adhering on the slide.

\subsection{Preparation lipids and cover glass}

To create the slides with parylene and lipids, cover glasses (5 mm diameter) is first coated with a thin layer of parylene. Then the slides are dipped into lipid solution, covering the whole surface in lipids. then the slides are dried, so lipids can settle on the surface. 

Two different lipids are used: DOPC and EGG-PC. DOPC is storaged in powder form. The first step is to solve the DOPC Powder in Ethanol (25 mg / 1 ml). Then, the solution is transferred into several small bottles. EGG-PC is shipped solved in chloroform. Two different ratios were used: 25mg / 1 ml and 10 mg / 1ml. the phioles were broken and then also transferred into several small bottles. small bottles were chosen because if solution is coating the threads of the cap, the bottle cannot be closed airtight anymore, leading to evaporation in the flask. By splitting it into multiple flask and using the solution, only one bottle with a small part of the solution is not airtight.

\subsection{solubility lipids}

Two different solubility experiments are proposed. The first is at room temperature to find solvents which work generally at higher temperatures. With the results, first solution which don't solve the lipids can be left out of the next experiment, as there are only limited baths available. The next one is at cryogenic temperatures to find solvents which also work at cryogenic temperatures.

These tests are conducted to find a fluid to solve a sacrificial layer out of lipids.

\subsubsection{at room temperature}

First, the potential solvents are picked. For that, the tested liquids needs to be save for humans in such way that no extractor hood is needed. This is needed because the following experiment does not fit under an extractor hood. The tested substances are 4-Methyl Pentene, 3-Methyl Pentene, 1-Pentene, Isopentane, 1-Propanol, Pentane and Ethanol. Each liquid is put in a separate bottle. Then the slides are prepared  as previously described.



\subsubsection{at cryogenic temperature}

\section{"finger"}

Funktionsweise FInger

Ablauf finger



\subsection{determining needed amount of glue}

First, the HFE used as glue was applied with a pincer. TO archieve this, the HFE is given in a cold bath at -140°C. This stops HFE from evaporating. The thickened hfe is now scooped with pincers on the tip of the finger. Though the correct amount is only determinable qualitative.

As an effort to determine the correct amount of HFE a pipette was used. ...



\subsection{temperature test}

\subsection{detaching ice from lipids}

\section{PDMS}

Why PDMS was chosen

\subsection{Preparation of PDMS samples}

All PDMS samples with different mixtures are prepared in the same way. The preparation starts with weighting out the needed amount of base coat and curing agent. The mixture is now stirred intensively. Then the mixture is placed in a vacuum bell for 30 minutes to gas out air bubbles. Meanwhile the cover glasses used as slides are cleaned with ethanol or isopropanol. Afterwards, the DPMS mixture is coat-spinned on the cover glass for 5 seconds with 300 rpm and then 120 seconds with 3000 rpm. Then the coated cover glasses are baked in the oven for 30 minutes exept a mixture of 1 base coat and 2 curing agent mixing ratio PDMS. For those 24 hours are needed, as it takes longer to harden and it will result in unwanted effects at plasma curing.

\subsection{Influence of plasma treatment on PDMS}

\subsubsection{setup}

\subsection{detaching ice from PDMS}

1:2 and 4:1

