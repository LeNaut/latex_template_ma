Das ist die Methode.

\section{Task and requirements}

In Application, samples (for example cells) are frozen inside an thin ice layer. Te sample can be stained with fluorescence and observed with cryo light microscopy. Also a sample can be prepared to study with cryo-transmission electron microscopy (cryo-TEM) . This allows to see samples in an hydrated state. This is only possible in cryo-TEM, as liquid water would evaporate in vaccuum \cite{Danino.2012}.

For sample preparation in cryo light microscopy and cryo-TEM, plunge-freezing is used \cite{Danino.2012} \cite{Faoro.2018}. This can be done either by hand or with a plunge-freezer. In both ways, the slide is first held by tweezers. then a 2 to 4 ml water drop including the sample is pippetted on the hydrophilic slide. The droplet spreads over the whole slide. Then the water droplet is blotted with a filter paper, creating a thin film of water which is evaporating fast. Then the slide with the sample is put inside of a cold liquid like liquid ethane. The cold liquid should not have the Leidenfrost effect, which makes a needed temperature drop of over 100°C in milliseconds possible. With this procedure, vitificated ice is formed without a crystal structure. A vitrificated sample is needed as ice crystals damage the samples and can disturb cryo light microscopy.

The main motivation for this master thesis is to find a way to use cryo light microscopy and cryo-TEM on the same sample. But currently, no slide is found which has all requirements to be used in plunge-freezing, cryo light microscopy and cryo-TEM. in plunge-freezing, a hydrophile surface is needed to archieve a thin ice layer. Additionally the thermal conductivity of the slide needs to be high for the steep temperature drop needed to create vitificed ice. For light microscopy, a transparent slide is not always required. But a goot thermal conductivity is advantageous as less energy is needed to keep the sample cool (WAS FÜR VORRAUSSETZUNGEN GIBT ES DA?). In cryo-TEM, the sample needs to be extremely thin and small. Additionally, only light elements should be used as heavier electrons are disturbing the image in cryo-TEM. As those requirements are very specific, a slide change could work around those issue.

Still, the slide change must be performed at -140°C to maintain the vitrificated state of the sample. Additionally as the first slide used for plunge freezing is hydrophilic, lifting the sample is not simply possible without designing a new layer. First, I investigated lipids for potential positive characteristics for a sacrificial layer or detaching it mechanically. then I am trying PDMS and use different mixture ratios and plasma curing to make mechanical removal easier.

\section{Phospholipids}

Phospholipids are the building block of membranes in nature. They own two long hydrophobic chains and a polar head. the membrane is formed by a bilayer with the hydrophobic parts showing inwards and the hydrophilic head pointing outwards. They are also natural detergents, as they can bind to hydrophobic DIRT and forming an emulsion, making them removable with polar liquids like water \cite{SriramaM.BhairiPh.D..}.

\subsection{Parylene}

 SOMETHINGSOMETHING?

\subsection{Preparation lipids and cover glass}

\subsection{solubility lipids}

\subsubsection{at room temperature}

\subsubsection{at cryogenic temperature}

\section{"finger"}

\subsection{determining needed amount of glue}

First, the HFE used as glue was applied with a pincer. TO archieve this, the HFE is given in a cold bath at -140°C. This stops HFE from evaporating. The thickened hfe is now scooped with pincers on the tip of the finger. Though the correct amount is only determinable qualitative.

As an effort to determine the correct amount of HFE a pipette was used. ...



\subsection{temperature test}

\subsection{detaching ice from lipids}

\section{PDMS}

Why PDMS was chosen

\subsection{Preparation of PDMS samples}

All PDMS samples with different mixtures are prepared in the same way. The preparation starts with weighting out the needed amount of base coat and curing agent. The mixture is now stirred intensively. Then the mixture is placed in a vacuum bell for 30 minutes to gas out air bubbles. Meanwhile the cover glasses used as slides are cleaned with ethanol or isopropanol. Afterwards, the DPMS mixture is coat-spinned on the cover glass for 5 seconds with 300 rpm and then 120 seconds with 3000 rpm. Then the coated cover glasses are baked in the oven for 30 minutes exept a mixture of 1 base coat and 2 curing agent mixing ratio PDMS. For those 24 hours are needed, as it takes longer to harden and it will result in unwanted effects at plasma curing.

\subsection{Influence of plasma treatment on PDMS}

\subsubsection{setup}

\subsection{detaching ice from PDMS}

1:2 and 4:1

