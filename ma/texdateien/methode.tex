% !TeX spellcheck = en_US

To detach the ice layer from the sample, two ideas are propose. The first proposal for detaching the ice layer is using mechanical force for separation. The second Idea is using a sacrificial layer which is dissolved at cryogenic temperatures.

\section{Detaching ice mechanically}

 To apply mechanical forces to the ice layer, a lifting assembly referred to as "finger" is used. To make lifting possible, the bottom layer is engineered to reduce the adhesion of the ice and the stability of the sample slide. Also, as the assembly is not used in similar work before, different variables are addressed and examined.  

\subsection{Assemblies used at cryogenic temperatures}

The finger is made of two main parts. The first part is metal rod with a slightly pointed tip (Fig. \ref{fig:querschnittfinger}). The rod is cooled with cold nitrogen gas. Near the tip, the rod is temperature controlled with a temperature sensor and a heater. The second main part is a 3D printed shell, containing the outer layer and routing of the cold gaseous nitrogen. The cold nitrogen gas is directed downwards around the metal bar in an inner mantle for cooling. then, the gas is directed upwards flowing through an outer mantle for additional cooling. Then the gas exits through the output.

\begin{figure}[hbt!]
	\centering
	\input{../images/ZeichnungFinger.pdf_tex}
	\caption{A cross-section of the "finger" assembly. The metal rod is cooled with cold nitrogen gas. the gas is routed from the inlet around the metal bar onto an outer layer to the outlet. The metal rod is temperature controlled by a temperature sensor and a heater. With HFE 7200 is applied to the tip, the finger can attach to a surface at cryogenic temperatures and apply force. }
	\label{fig:querschnittfinger}
\end{figure}

The Cold nitrogen gas is supplied by a liquid nitrogen tank. Strong heaters are placed inside the tank to evaporate the liquid nitrogen. The volume is rapidly expanding at evaporation, resulting in fast gas flow. the cold nitrogen gas is routed by \SI{6}{\milli\meter} pneumatic tubing. The inlet of the finger is connected to the liquid nitrogen tank. At the inlet and outlet of the finger, Festo connectors are mounted to allow easy dis- and reconnecting. the outlet tubing exhausts the cold nitrogen into the atmosphere.

The finger is  mounted on three stages. additionally the stages are mounted on a track. the three stages allow fine adjustment of the finger position in X,Y and Z axis. Also, when the finger is attached to a surface, force can be applied by moving the stages in either direction. The Finger can also be moved along the track. In use, the assembly is clamped down on the track to prevent movement. when not in use, the assembly can be moved on the track to allow easy access of the area below the "finger".

GRAPHIC STAGES NOCH HINZUFÜGEN

On the tip of the finger, Hydrofluorether (here HFE 7200) is applied. HFE is an oil typically used as an cryoimmersion fluid \cite{Faoro.2018b}. Besides that, it has temperature dependent abilities. At freezing point, HFE reaches a viscous state before reaching a firm solid state. this temperature dependency is used to first apply the HFE at higher temperatures with low viscosity and pull on the sample at low temperatures with high viscosity.

In the beginning a smaller bath is used (Fig. \ref{fig:KleinesBad}). The Small bath contains an elevated floor as work surface. embedded in the work surface are indents which are used as container holder. those container allow transport and long term storage of samples. Three elevated baths are installed above the work station. They are used for temperature controlling and containing other Liquids or tools. Also a Haven for a shuttle system is installed. The small baths and the haven are elevated over the second floor with an insulating layer, so a temperatures over \SI{-195.8}{\degreeCelsius} can be regulated. The liquid nitrogen is filled over the work surface, but not over the insulating layers to allow temperature controlling. The whole bath is insulated by Nitrogen gas flowing inside the 3D-printed shell of the bath. Then the Nitrogen gas is expelled from the brim, pointing from the outer edge radially to the rotation axis. In that way, the nitrogen gas separates the damp air in the room with the dry nitrogen gas inside the bath. Therefore ice formation inside the bath is inhibited.

This small bath and later the big bath are also used in combination with the finger. The finger is positioned over the shuttle docked in the haven. The sample is fixed in the shuttle. The position of the finger over the shuttle is manipulated with stages. The warm Nitrogen gas barrier also keeps the finger tip ice free.

The usage of the small bath in combination of the finger has limitations: First, the space is small. The finger can be moved along the track, but the space left still limits work freedom with pincers. Additionally, the smaller baths are not needed when using the finger, using up more space. Also the Shuttle needs to be tilted in a specific angle so docking and undocking of shuttle in the haven is possible. The work flow also allows only one shuttle at once, limiting throughput. Also liquid nitrogen needs to be refilled often since the bath can only hold a smaller volume.

\begin{figure}[hbt!]
	\centering
	\begin{overpic}[width=10cm]{SmallBath}
		\white
		\put(40,25){\vector(1,1){10}}
		\put(40,25){\makebox(0,0)[r]{shuttle haven}}
		\put(20,45){\vector(1,0){15}}
		\put(20,45){\makebox(0,0)[r]{tool bath}}
		\put(73,35){\vector(-1,1){10}}
		\put(73,35){\makebox(0,0)[l]{ethanol bath}}
		\put(73,27){\vector(-1,1){10}}
		\put(73,27){\makebox(0,0)[l]{HFE bath}}
		%\put(49,76){\vector(-0.15,-1){3.4}}
		%\put(51,76){\vector(0.15,-1){3.4}}
		\put(48,65){\vector(-0.25,-1){2.8}}
		\put(52,65){\vector(0.25,-1){2.8}}
		\put(50,65){\makebox(0,0)[b]{container holder}}
		%\put(75,78){\vector(-2,-1){10}}
		%\put(76,78){\vector(-1,-2){4.5}}
		\put(75,72){\vector(-1,0){10}}
		\put(79,68){\vector(-1,-1){3.5}}
		\put(75,72){\makebox(0,0)[lt]{gas outlets}}
		\put(75,72){\makebox(0,0)[lb]{nitrogen}}
		\put(30,62){\vector(1,-1){10}}
		\put(30,62){\makebox(0,0)[rb]{work surface}}	
	\end{overpic}
	\caption{Small bath used for sample preparation. This bath is combined with the finger. A major drawback of use with the finger is the lack of space inside the bath. To solve this issue, a bigger bath is constructed.}
	\label{fig:KleinesBad}
\end{figure}

During this master thesis, a second bigger bath is build (Fig. \ref{fig:GroßesBadMitFinger}). In general, the structure is similar. It also has an elevated floor as a work surface. The work surface is fabricated out of two plates screwed together. Indents are formed by holes in the upper plate. No baths are installed, but the space is planned in for later addition. The work surface is held in place by 3D printed holders which are fixed to the brim of the metal bath. Also two harbors can be mounted for parallel work on two separate shuttles. The harbors are screwed on an aluminum block. the aluminum block is temperature regulated. between the aluminum block and the work surface, a 3D-printed insulating layer is placed. Also both harbors can be mounted either flat or in an angle, depending of the 3D printed layer. The Bath is insulated with styrofoam and a rim with holes for warm nitrogen gas is placed on top. The holes are places along the inside of the longer perimeter, so the stream covers the whole area with minimal turbulence. This also keeps the inside ice free.

\begin{figure}[hbt!]
	\centering
	\begin{overpic}[width=10cm]{BigBathWithFinger}
		\white
		\put(24, 35){\vector(1,1){10}}
		\put(24, 35){\makebox(0,0)[t]{shuttle haven}}
		\put(46, 31){\vector(-1,1){10}}
		\put(46, 31){\makebox(0,0)[t]{temp. controlled aluminum block}}
		\put(25, 75){\vector(1,0){10}}
		\put(25, 75){\makebox(0,0)[r]{"finger"}}
		\put(15, 19){\vector(-1,-1){10}}
		\put(15, 19){\makebox(0,0)[l]{inlet warm nitrogen gas}}
		\put(50, 15){\vector(0,1){5}}
		\put(50, 15){\makebox(0,0)[t]{hole for refilling}}
		\put(40, 85){\vector(-1,1){5}}
		\put(40, 85){\makebox(0,0)[l]{cold nitrogen tubing}}
		\put(15, 58){\vector(0,-1){15}}
		\put(16, 58){\vector(0,-1){10}}
		\put(16, 58){\makebox(0,0)[b]{gas outlets nitrogen}}
		\put(32, 52){\vector(0,1){5}}
		\put(32, 52){\makebox(0,0)[t]{container holder}}
		\put(17, 88){\vector(0,-1){5}}
		\put(17, 88){\makebox(0,0)[b]{connection "finger" to stage}}
	\end{overpic}
	\caption{Big bath with finger assembly. Inside samples are prepared as well as the finger is used for applying force on the sample.}
	\label{fig:GroßesBadMitFinger}
\end{figure}

To take pictures of the sample, an inverted microscope is modified for cryo microscopy (Fig. \ref{fig:Mikroskop}). A box with a metal core is fixed to the bottom of the stage. an outer frame an the top of the box is temperature controlled. The box is supplied with cold nitrogen gas to cool the iron core. A haven is fixed to the iron core and placed over the objective. The haven is temperature controlled to \SI{-140}{\degreeCelsius} to make remains of HFE liquid and distinguishable. to keep the light path between haven and objective clear from ice and fog, a component reffered to as "glasses" is used(HIER REFERENZ ZUR ZEICHNUNG VON DER BRILLE?). It contains two parallel glass slides. Warm nitrogen gas is routed beween the glass slides and the lower glass slide and objective. Additionally, a filter is used for examining fluorescent samples.

The Shuttles are also used in cryo light microscopes. The Microscope used for cryotemperatures have an additional box installed, routing Cold nitrogen gas underneath a harbor, where the sample is placed. Heaters are placed around the box and under the harbor to archieve a constant temperature. On top of the Harbor, warm Nitrogen is blown so no ice is forming inside the optical path.

\begin{figure}[hbt!]
	\centering
	\begin{overpic}[width=10cm]{Microscope}
		\put(24, 45){\vector(1,0){10}}
		\put(24, 45){\makebox(0,0)[r]{cooled box}}
		\put(15, 50){\vector(1,0){10}}
		\put(15, 50){\makebox(0,0)[r]{frame}}
		\put(35, 25){\vector(1,0){5}}
		\put(35, 25){\makebox(0,0)[r]{objective}}
		\white
		\put(52, 33){\vector(-1,0){10}}
		\put(52, 33){\makebox(0,0)[l]{"glasses"}}
		\put(55, 52){\makebox(0,0)[c]{cold nitrogen gas supply}}
		\put(50, 75){\makebox(0,0)[b]{stage}}
		\put(59, 40){\vector(-1,0){10}}
		\put(59, 40){\makebox(0,0)[l]{gas outlet}}
		\put(32, 35){\vector(1,0){10}}
		\put(32, 35){\makebox(0,0)[r]{access shuttle}}
		
		
	\end{overpic}
	\caption{Modified microscope for cryo microscopy.}
	\label{fig:Mikroskop}
\end{figure}

The shuttle allows easy transportation of the sample without limiting the access to the sample. The sample is clamped down by a brace, which is also reffered to as "window". The "window" is fixed with two screws to the shuttle, holding the sample in place. the "window" gives access to the top side of the sample with a center hole for microscopy and "finger". A long rod with a thread at one side and a temperature insulated knob on the other side is used to screw into the sample. This allows easy transport between bath and microscope. 

\begin{figure}[hbt!]
	\centering
	\input{../images/Zeichnung_shuttle.pdf_tex}
	\caption{Shuttle for transporting a sample between bath and microscope. The "window" is clamped down by screws to the copper shuttle, holding the sample in place. Through the hole in the window, microscopy is done on the sample. The "finger" fit also through the "window" for pulling. For transport, a rod is screwed into the copper shuttle.}
	\label{fig:shuttle}
\end{figure}

On the other hand, small container with space for three $\varnothing$\SI{5}{\milli\meter} samples are used (Fig. \ref{fig:transportbox}). These are 3D Printed, modified version of other containers. a cap which is identical with the other versions of container is screwed on top with a special pencil. Inside the bath, the container fit into the indent. the cap is screwed next to the container, holding it into place. Also the container can be stored in the same container of the other version. This allows long term storage of samples with \SI{5}{\milli\meter} diameter.


\begin{figure}[hbt!]
	\centering
	\input{../images/Zeichnung_transportbox.pdf_tex}
	\caption{Small transport box for three samples. a cap is screwed onto the central thread. (WAS ZUM DECKEL SCHREIBEN). These boxes fit into storage units for long time storage.}
	\label{fig:transportbox}
\end{figure}

\FloatBarrier

\subsection{Process}
\label{section:Process}

The goal is to mechanically lift a piece or the whole ice layer from the slide it is frozen to. In the whole process, the ice stays in the vitrified state. The sample is prepared in the bath. The "finger" is used to attach and pull on the ice layer with HFE. Also the microscope is used to take pictures of the sample without heating up the sample.

To try out the detachment with the "finger" in a repeatable manner, a core process is established for orientation. The sample is prepared and put in the bath. The copper shuttle part is placed in the harbor. With HFE, the sample is placed in the middle of the shuttle. The HFE helps the sample to stay in place before fixation. The window brace is placed on top and screwed down. The now prepared shuttle is transported quickly to the microscope. When the transfer is not possible within seconds, the shuttle is placed in a portable container with liquid nitrogen. After microscopy, the sample is placed into the bath unter the cooled finger. 

First, if not already done, cool down the "finger" to \SI{-140}{\degreeCelsius}. HFE is applied to the tip. The "finger" is lowered onto the sample while correcting the position with the stages until the HFE contacts and spreads over the sample. The temperature is reduced to \SI{-160}{\degreeCelsius} and waited until the sample and finger is cooled down. When the temperature is reached the finger is pulled up by turning the stage until detachment. Then the shuttle is transported to the microscope and the sample is analyzed. 

When detachment is successful, the ice piece hanging on the "finger" is placed on another shuttle. This is done by lowering the finger onto the new shuttle and raising the Temperature to \SI{-140}{\degreeCelsius}.

To collect first insight, samples with parylene and lipids (as described in Section \ref{section:metodeLipide}) are used first. different variables are determined which could significantly influence successful detaching. Then the different variables are examined with experiments to improve the reliability of the "finger". The different variables found in this thesis are discussed in the following sections.

\subsection{Volume of Hydrofluorether on "finger"}

Using the correct amount of HFE is important for higher repeatability. Too little HFE will not bind to the finger and sample. too much HFE results either in a thicker layer, or the HFE flows between "window" and sample.
as the cohesion of the HFE is considerably lower than the sample and the finger, a thick layer is prone to breaking before the ice layer. HFE under the "window" clamp will redirect a part of the force, making the sample more stable. This makes detaching also harder.

In the beginning, the HFE was applied with a pincer.The HFE is given in a cold bath at \SI{-140}{\degreeCelsius}. HFE gets more viscous at colder temperatures and is evaporating much slower. The thickened HFE is now scooped with pincers onto the tip of the "finger". This method is not used later, as measuring the volume of HFE when applying is not possible.

As an effort to determine the volume of HFE a pipette is used. The HFE is pipetted at room temperature onto the cold finger. When HFE is applied onto the desired surface, around \SI{4}{\micro\liter} has already evaporated. Also, only HFE on the flat surface facing the sample is usable.

Another way to determine the Volume of HFE on the "finger" is by analyzing pictures of the "finger" with HFE. The volume is calculated with the contact angle on the "finger" and the area covered in HFE. With the knowledge if the HFE spreads too much or not being attached properly to the sample, a range can be given where success is more likely. Still, other factors like temperature or the gap between sample and "finger" can influence the result.
%In the first experiment, I dosaged three different volumes of HFE onto the tip of the finger with the pipette. The amount on the finger did not correspond to the amount of HFE. So other variablilities like the time between loading and unloading and hitting the correct spot on the finger is more relevant than the amount of HFE used.

%To still determine the correct glue amount, two pictures representing the lowest and the highest usable HFE amount was picked. then the volume is calculated. this can be used as reference for future work for dosaging the right amount of HFE.



\subsection{Temperature}

Ethoxynonafluorobutane, also called HFE 7200, is used throughout all experiments. It has a freezing point of \SI{-138}{\degreeCelsius}. Below the freezing point, HFE gets increasingly viscous. at a certain point, the hfe gets brittle and cracks start to form by decreasing temperature. Between the freezing point and the point of cracking, HFE is usable as glue. With a temperature near the freezing point, HFE can spread over the surface. At lower temperatures, HFE gets hard enough to hold the finger to the upper ice layer.

The temperature regulation of the finger has three modes: First the "unglue" mode which regulates the Temperature to \SI{-140}{\degreeCelsius}. HFE has a low viscosity, which allows the application of HFE. Also detachment without transferring force is possible. Second, in "glue" mode the shuttle and the finger are cooled to \SI{-160}{\degreeCelsius}. HFE hardens and force can be applied to detach the ice layer. Third, "thaw" cleans the finger by heating the tip to \SI{20}{\degreeCelsius}, evaporating everything stuck on the "finger". 

As HFE is getting harder with temperature, lower temperatures in the "glue" state could allow higher forces on the ice layer. Still, the point where HFE starts to crack will decrease the tensile strength. To test this hypothesis, HFE is examined at temperatures until \SI{-170}{\degreeCelsius}.

\subsection{Direction of force}

The direction of force can increase the likelyhood of detachment. Tensile mode and shear mode can be applied by the finger. Tensile mode is the easiest to apply with the finger. Also HFE is able to withstand some tensile forces. But for separating layers, this mode could take more force than applying tensile forces, depending on the bottom layer. Still, HFE stability to shear stress is not known. Additionally, the sample is clamped down to the top. This does not allow the ice surface to slide off without breaking.

In some experiments, the shuttle is tilted around 15 degrees for easier access to the shuttle. The finger is also tilted so the tip surface is parallel to the sample. To apply force, the finger is pulled by stages in either X, Y, or Z direction. For each direction, the stress is split into tensile and shear stress. In Z direction, mostly tensile stress is applied (Fig. \ref{fig:tensilevsshear} (a)). in X direction, mostly shear stress is applied (Fig. \ref{fig:tensilevsshear} (b)). In Y direction, only shear stress is applied, but will result in shattering of the ice layer, probably taking additional force. For this reason, only X and Z direction are tested.

\begin{figure}[hbt!]
	\centering
	\input{../images/Zeichnung_finger_Tensile_vs_Shear.pdf_tex}
	\caption{Tensile vs Shear mode}
	\label{fig:tensilevsshear}
\end{figure}

\subsection{ice structure}

DER TEIL BRAUCHT NOCH ARBEIT!!!!!!!

To lift off a piece of the ice layer, the ice layer must be broken in some way. thicker ice layers are expected to be harder to break than thinner ones due to the bigger cross section. Also amorphic vitrified ice is expected to be more stable than crystallized ice.

Initially, to save time for experiments, the samples are freezed by Hand in liquid nitrogen, as described before. However, the ice layers are less consistent compared to plunge freezing, resulting in mostly thicker ice layers compared to plunge freezing. also as the sample is frozen in liquid nitrogen, the leidenfrost effect is inhibiting the formation of vitrified ice.

In experiments, the used glass slide and freezing method does not produce vitrified ice. Therefore the influence of vitrification is not observed. The thickness of the ice is also not measured directly. 

To compare the influence of hand freezing and plunge freezing, results of lifting off samples frozen with both methods are compared. No other factors are varied in those experiments. In the end, hand freezing and plunge freezing did not make a difference. Therefore, hand freezing was also applied in future experiments, as this effect is determined as neglegtable compared to other factors.

SAMPLE INTEGRITY

\subsection{positioning}
\label{section:positioning}

Before attaching the "finger" to the sample, the "finger" is positioned with the stages. At positioning, two different errors can occur: first, the finger is incorrectly positioned over the sample. the HFE is partially on the "window" brace. The second error is a too small or too big gap between "finger" and sample. 

An incorrectly positioned finger is expected to be a major error. Depending on the overlap, the force is transferred mostly to the brace than to the sample. This will lead to an unsucessful detachment. With the setup of three stages, the positioning is easily corrected. Alternatively, a "window" brace with a bigger central hole makes positioning easier.

The gap between sample and "finger" is sometimes hard to estimate. The HFE volume can influence the preferred gap size to avoid pressing the HFE between sample and "window". On the other hand, HFE is shrinking between temperatures used at attaching and pulling.

 (HIER WEITERSCHREIBEN ODER BEI ERGEBNISSE??) 

\section{PDMS}

PDMS is a polymer used in coatings for passive deicing. It is hydrophobic and has a low surface energy. Also it can be coat spinned into a thin layer to form the thin coat. Also it is widely available and tunable. To use PDMS in plunge freezing, the PDMS is plasma activated. The requires a Plasma generator.

To create different PDMS mixtures, Dowsil Sylgard 184 Silicone elastomer is used\cite{DOW.}. The PDMS kit has two components. The base coat component is highly viscious, whereas the curing agent is liquid. The Specified mixture ratio is 10 base coat to 1 curing agent in weight (10:1). In some applications, other mixture ratios are used and additives are added for tuning PDMS. In my research, I focussed on tuning the mixture ratio of base coat to curing agent. 

PDMS properties are also temperature dependent. In \cite{Zhang.2020}, multiple characteristics are determined for cryogenic temperatures. The PDMS is prepared with the standard mixture ratio of 10:1. The compressive strength increases with lower temperatures until \SI{-123.15}{\degreeCelsius}. At this temperature, the compressive strength reaches a maximum of \SI{224.50}{\mega\pascal} in average. At lower temperatures, the PDMS gets brittle. At \SI{-150.15}{\degreeCelsius} PDMS has a compressive strength of \SI{106.99}{\mega\pascal}.

The "finger" requires a temperature of \SI{-160}{\degreeCelsius}. Therefore, the temperature drops enough to make PDMS of 10:1 mixture ratio brittle. Generally, a brittle PDMS surface helps to detach the ice layer. Tensile forces loosens the PDMS under the Ice. The PDMS layer breaks within itself, loosening the ice layer from the rest of the sample. Still, an extreme brittleness is needed, which may not be reached. With PDMS of other mixture ratios, the brittleness could change, including the temperature dependency.

\subsection{Plasma surface treatment of PDMS}

Plasma curing is commonly used as PDMS treatment. For example, it is used to bond two PDMS surfaces together \cite{Borok.2021}. It is also used for increasing wettability and adhesion. Plasma treatment is changing the chemistry of the polymer chains on the surface. Charged Oxygen Ions are deposited on the surface. these Ions make the surface temporarily hydrophile and increasing water adhesion. The Ions change the structure of the PDMS is permanently by oxidation. In some cases, cracks form as the surface oxidizes to a silica like form.

In \cite{Owen.1994}, the influence of different gasses on PDMS plasma treatment is examined. Oxygen, Nitrogen, Argon and Helium are compared on the effect on wettability, adhesion and cracking. Thin PDMS sheets are used with unknown composition. They found similar results between gasses. All gasses produced a thin and brittle surface with cracks and high wettability. Based on these results, used gas is not a significant factor for plasma activation. Therefore using only air (mainly a mixture of nitrogen and oxygen) is sufficient to determine the effect of plasma treatment.

%&The effect of Plasma activation between mixture ratios is mostly unknown. At mixture ratios above 50:1 base coat to curing agent, no significant differences between plasma activation is observed \cite{Ohishi.2017}. 

In \cite{Ohishi.2017}, the influence of plasma treatment on Mixture ratios of 50:1 to 100:1 is described. A thin layer of those PDMS mixtures is put onto a preformed PDMS piece with lower mixture ratio. The preformed piece is used to apply shear stress to the surface by stretching the lower PDMS form piece. Therefore only tensile force are examined. It shows no significant difference between described mixture ratios. It also shows that higher plasma treatment leads to more brittle surfaces, reducing the force needed to break the PDMS Layer by $90\,\%$.

Also known are the adhesion forces on PDMS from 1:3 to 50:1 mixture ratio. In \cite{IbanezIbanez.2022}, PDMS is mixed in different weight ratios. The mixture is given into a mold and afterwards fully cured. An ice block is frozen on top of the PDMS. Using a pulling machine the maximum tensile and shear forces for detaching ice from PDMS are determined. Results show that mixture ratios 10:1 to 1:3 have significant lower adhesion forces on ice. At for example at 2:1, the shear mode is just under \SI{20}{\kilo\pascal} and the tensile mode around \SI{30}{\kilo\pascal}. 

Still, the effect of plasma activation under 50:1 is unknown. The plasma activation has two effects: first, it can increase the adhesion on ice to PDMS. second, the PDMS changes its structure, which could increase as well as decrease the strength of the PDMS layer. In the following, the effect of plasma activation on PDMS are examined. This is done at room temperature to speed up the process.

\subsection{Preparation of PDMS samples}

All PDMS samples are prepared in a similar way. The preparation starts with weighting out the desired amount of base coat and curing agent. The mixture is stirred intensively. The mixture is placed under a vacuum bell to gas out all air bubbles from stirring. Meanwhile the glass covers are cleaned with ethanol or isopropanol and dried. Afterwards, the PDMS mixture is coat-spinned onto the glass covers. Then the coated cover glasses are baked in the oven.

For 1:2 base coat to curing agent weight ratio, the PDMS mixture is lightly viscious. A vacuum of \SI{30}{\minute} is drawn for degassing. A coat spinning time of \SI{120}{\minute} at \SI{3000}{\rpm} results in a smooth surface on all used slides. The baking time of at least \SI{24}{\hour} by \SI{80}{\degreeCelsius} is needed. For shorter baking times, plasma treatment has a slighlty different effect. Normally, touching a treated area will neutralize the effect of plasma treatment like hydrophility only locally on the touched surface. But here, touching the surface leads to the complete neutralization of plasma treatment when touching. In this work, the effect is undesired.
 
For 4:1 base coat to curing agent weight ratio, the PDMS mixture is more viscious. The Vacuum and coat spinning are the same for \SI{30}{\minute} vacuum and \SI{3000}{\rpm} for \SI{120}{\minute}. But a baking time of \SI{30}{\minute} at \SI{80}{\degreeCelsius} is already sufficient to harden the PDMS. 

For 50:1 base coat to curing agent weight ratio, the PDMS mixture is as viscious as the base coat. A longer vacuum of \SI{1}{\hour} is needed to air out all bubbles. The coat spinning speed and time is increased to \SI{3500}{\rpm} for \SI{180}{\minute} (NACHSCHAUEN). Also a longer baking time of \SI{20}{\hour} at \SI{80}{\degreeCelsius} is needed to harden the PDMS. 

In the coat spinning setup, each slide is coat spinned in succession of each other. Each coat spinning process needs 2 to 3 minutes. This results in a time consuming process. To speed up the process, rectangular cover glass with 20x20 and 24x40 are coat spinned. afterwards the cover glass is split in multiple smaller pieces. 

The process is the same described as before for each PDMS mixture ratio, but with some adjustments. Before coat spinning, the glass is scratched with a diamond pencil. The scratches in the glass are weak points for breaking out smaller parts. Tape is put over the scratched glass. The glass is put into the coat spinner with the taped side facing down. The PDMS is coat spinned onto the exposed glass side. The baking process is as previously described. After baking, the glass is broken into smaller pieces. This is done by dragging the attached tape over a table edge. Then the glass is fixedto the table with PDMS facing up. The PDMS is covered with (????) to keep the PDMS clean. As the sticky side of the tape is not facing the table, the tape ends are additionally taped to the Table. Then smaller sample pieces can be broken off, The pieces are loosened from the tape by forcing a flat pincer between glass and the tape it is attached to.

This method has advantages as well as drawbacks to coat spinning small glass pieces separately. First, time is won by coat spinning, as one big glass can be split into several smaller ones. The Sample can stay fixed at the table without risking damaged. The samples can be broken off shortly before the experiment. But with hand scratching, only irregular and rectangular shapes can be won out of the bigger glass piece. also there is a significant loss of samples. This is by one part by breaking the sample. Some cracks do not follow the scratch, which leads to too small samples. Also in the process of loosening a piece cracks can form, making the piece too small for use.

In the end, these two different fabrication methods are used for different experiments. The small round cover glasses are used in combination with the finger. The regular pieces are easier set up with the shuttle. The glass pieces broken out of a big cover glass are used under the pulling machine at room temperature. In this setup, fluctuations in size of the glass pieces are easier to handle and the clamp has more flexibility.

\subsubsection{Setup of the pulling machine}

To test the tensile strength of different PDMS mixtures and the effect of plasmacuring, a Pulling machine (NAME RAUSFINDEN)is used. On the top part, two load cells are installed (Fig. \ref{fig:PullingMachineSetupBigPic}) The upper load cell is rated for \SI{2}{\kilo\newton}(ÜBERPRÜFEN). the lower load cell is rated for up to \SI{100}{\newton}. As the upper one is extremely stiff and the forces are \SI{<<100}{\newton}, the upper sensor is assumed as inflexible. On both the upper and lower part, two clamps are fixed onto the machine. On the bottom clamp a 3D-Printed stage is used for fixing on the sample.

\begin{figure}[hbt!]
 	\centering
 	\begin{subfigure}[]{0.45\textwidth}
 		\centering
 		\begin{overpic}[width=6cm, height=9cm]{AufbauPullingMachine}
 			\white
 			\put(45, 82){\vector(-1,0){10}}
 			\put(45, 82){\makebox(0,0)[bl]{\SI{2}{\kilo\newton}}}
 			\put(45, 82){\makebox(0,0)[tl]{Load cell}}
 			\put(20, 55){\vector(1,0){10}}
 			\put(20, 55){\makebox(0,0)[br]{\SI{100}{\newton}}}
 			\put(20, 55){\makebox(0,0)[tr]{Load cell}}
			\put(20, 35){\vector(1,0){10}}
 			\put(20, 35){\makebox(0,0)[r]{top clamp}}
 			\put(45, 20){\vector(-1,0){10}}
 			\put(45, 20){\makebox(0,0)[bl]{bottom}}
			\put(45, 20){\makebox(0,0)[tl]{clamp}}
			%\put(21, 29){\vector(1,0){10}}
			%\put(21, 29){\makebox(0,0)[r]{Fig. \ref{fig:PullingMachineSetupZoomedPic}}}
 		\end{overpic}
 		\caption{}
 		\label{fig:PullingMachineSetupBigPic}
 	\end{subfigure}
 	\begin{subfigure}[]{0.45\textwidth}
 		\centering
 		\begin{overpic}[width=6cm, height=9cm]{AufbauPullingMachineCloseup}
 			\white
 			\put(30, 78){\makebox(0,0)[c]{Top clamp}}	
 			\put(30, 14){\makebox(0,0)[c]{Bottom clamp}}
 			\put(21, 56){\vector(1,-1){10}}
 			\put(21, 56){\makebox(0,0)[r]{Stamp}}
 			\put(33, 26){\vector(0,1){10}}
 			\put(33, 26){\makebox(0,0)[t]{Stage}}
 			\put(45, 51){\vector(-1,-1){10}}
 			\put(45, 51){\makebox(0,0)[l]{Brace}}
 			\put(43, 39){\vector(-1,0){10}}
 			\put(43, 39){\makebox(0,0)[l]{Screw}}
 			\put(21.5, 40.5){\vector(1,0){10}}
 			\put(21.5, 40.5){\makebox(0,0)[br]{UV glue}}
 			\put(21.5, 40.5){\makebox(0,0)[tr]{on sample}}
 			
 		\end{overpic}
 		\caption{}
 		\label{fig:PullingMachineSetupZoomedPic}
 	\end{subfigure}
 	\caption{Setup for tensile tests of a thin PDMS layer. The Sample is placed on the stage. A brace is screwed over the sample to hold it in place. A stamp is fixed to the top clamp. The UV Glue is applied between stamp and sample and cured with 3 minutes UV Exposure. With the \SI{100}{\newton} load cell, the force is measured over distance pulled. The load on the \SI{2}{\kilo\newton} load cell is neglectable as the forces measured are \SI{<<100}{\newton}.}
 	\label{fig:PullingMachineSetup}
\end{figure}

On the bottom, a 3D-Printed stage is clamped (Fig. \ref{fig:PullingMachineSetupZoomedPic}). The stage has four holes with threads for screws . Between the threads, the sample is placed. A brace similar to the "window" is fixed with screws onto the sample. A rectangle shape hole gives access for the UV-glue and the stamp. 

The Process of a pull test starts with clamping a new stamp on the top. Then the sample is plasma treated. After treatment, the sample is quickly transported to the pulling machine and fixed to the stage with screws. The stamp is aligned to the sample. When alignment is finished, the UV glue is given onto the stamp. the top is lowered on the sample. The UV glue is cured with UV radiation for \SI{3}{\minute}. After gluing, \SI{3}{\minute} are waited until forces settle resulting from curing and heat. Then, the pulling machine pulls with constant speed and measures the stress strain curves. The process is ended manually. The now separated stamp and sample are stored for analyzation under a microscope. To calculate the maximum tensile stress, the area of the glue on the sample which is still attached to the stamp is measured.

\subsection{detaching ice from PDMS}

Now, different PDMS mixture ratio with different plasma activations are tested at cryogenic temperatures. With previous setup, plasma activation of 1:2 mixture ratio PDMS resulted into a more stable layer with more adhesion. Also 50:1 is tested with high plasma activation. The tensile stress of 50:1 mixture ratio is \SI{60}{\kilo\pascal} \cite{IbanezIbanez.2022}.  \cite{Ohishi.2017} suggests that strong plasma activation over \SI{3}{\minute} results of a decrease of adhesion strength of around$90\,\%$. Additionally, 4:1 mixture ratio with high plasma activation is tested.

Two different setups are used. First, the setup with "finger" and "bath" described in section \ref{section:Process} is used. Second, for additional data, a modified version of the setup with the pulling machine is used. This allows the measurement of the actual tensile strength on the sample. 

\subsubsection{Setup pulling machine for cryogenic tests}

To allow tests at cryogenic temperatures with the pulling machine, the setup is modified to allow fitting a bath and the "finger" to the pulling machine. The big bath is used with a flat harbor. An outer frame is used and modified to hold the bath. The bottom clamp is removed and the Bath is fixed over the bottom attachment. The top part is used with the same load cells, but with a bigger clamp. The "finger" 3D printed shell modified with a flat outer profile for clamping and clamped onto the top. \SI{90}{\degree} festo tubing is used to connect the cold nitrogen gas supply. A power source and a temperature regulator is again used. 

BILD

Compared to the "finger" setup with stages, the finger only moves up and down. The Alignment is done by moving the bath by loosening the screws on the frame. Also new "window" parts are waterjet cut. The new "window" has a bigger central hole for easier alignment.

As the pulling mashine is set up in a separate room, the samples are prepared in the laboratory next to the microscope. The small bath is used for sample preparation. The process is the same as all detachment trials with the finger. 

\section{Phospholipids}
\label{section:metodeLipide}

Phospholipids are the building block of membranes in nature. They are made of two long nonpolar carbon chains and a polar head. A membrane is a bilayer of Phospholipids with the hydrophobic carbon chains oriented inwards and the hydrophilic head pointing outwards. They are also natural detergents, as they can bind to hydrophobic waste forming an emulsion, making them removable with polar liquids like water \cite{SriramaM.BhairiPh.D..2001}.

Phospholipids are solvable in some common solvents. To apply Phospholipids, a slide is dipped into the solvent. When the solvent dries out, the lipids are binding to the surface of the slide, forming a layer with varying homogeneity. this layer can be removed with the same solvents. If the ice layer is frozen onto the lipid layer, the lipids can be solved at cryogenic temperatures, detaching the ice. But to solve this layer, a high solubility at cryogenic temperatures is required. 

% as the surface of the sacrifical layer is only on the edge of the sample.!

\subsection{Parylene}

Parylene is a hydrophobic polymer used as a coating to repel particles, including water and ice. Parylene is also biocompatible and used in medicine and biology (QUELLE?).

Parylene is not usable without a second layer on top. Parylene hydrophobicity does not allow water to spread during plunge freezing. with plasma activation, the surface is now hyrdophilic, but ice adheres to the parylene too strong to mechanically detach. also parylene cannot be solved with a solvent as a sacrificial layer.

For this reason, lipids are used in combination with parylene (Fig. \ref{fig:sacrificial layer}). The hydrophobic chains of the lipids adhere to the parylene. The polar head allow water to spread evenly over the surface. Solving the lipids with a solvent will detach the ice layer from the slide. Parylene additionally prevents (re-) attachment through holes in the lipid layer. 

\begin{figure}[hbt!]
	\centering
	\input{../images/Zeichnung_Layer_Lipide_parylene.pdf_tex}
	\caption{Layers of a Sample. The Lipid layer is used as a sacrificial layer. To reach the layer with a solvent, the only contact surface is to the edge. To get a fast and reliable process, a solvent with high solubility is needed.}
	\label{fig:sacrificial layer}
\end{figure}


\subsection{Preparation of lipid coated slides}

To create the slides with parylene and lipids, a cover glass (\SI{5}{\milli\meter} diameter) is used as base. the cover glass is coated with a thin layer of parylene. The coated cover glass is dipped into lipid solution. The cover glass is dried, leaving behind a lipid layer. The prepared slides are then used in plunge freezing.

Two different kind of lipids are used: DOPC and EGG-PC. DOPC is storaged as a powder. The DOPC powder is solved in Ethanol ($25\,\si{\milli\gram}/1\,\si{\milli\liter}$ lipid to solvent) for application. 
EGG-PC is shipped solved in chloroform in two different ratios: $25\,\si{\milli\gram}/1\,\si{\milli\liter}$ and $10\,\si{\milli\gram}/1\,\si{\milli\liter}$. The solution is shipped in phioles.

The lipid solution is transferred into several small bottles. small bottles are chosen because solution forms a lubrication film on the thread of the lid. this prevents the lid from closing airtight. This leads to evaporation of the solvent over time, making the bottle unusable. In the coating process, solution often drops onto the threads, making a bottle only usable in one coating session. By splitting the solution into multiple flask, more slides can be covered from one batch of solution.

\subsection{solubility lipids}

These tests are conducted to find a fluid to solve a sacrificial layer out of lipids. (BEDINGUNGEN SACRIFICIAL LAYER? ODER KAM DAS SCHON?)

Two consecutive solubility experiments are proposed. The first experiment is conducted at room temperature. the aim is to find solvents with high solubility at room temperature. the candidates with high solubility are then tested in the next experiment at cryogenic temperature. the aim is now to find solvents with also high solubility at cryogenic temperatures. The first experiment is conducted as there are only three baths available at cryogenic temperature. therefore the throughput for experiments is limited.

\subsubsection{at room temperature}

The Solvent are chosen based on availability, freezing point and safety. The solvents are all readily available in the laboratory. Some were ordered before the test. Also all chosen solvent are save to use in a well ventilated room. The followup experiment cannot be conducted under a extractor hood as too much space is taken up with the experiment. Also the solvent or solvent mixture needs to stay liquid at around \SI{-140}{\degreeCelsius} to assure that the ice layer on top stays vitrified. The tested substances are 4-Methyl Pentene, 3-Methyl Pentene, 1-Pentene, Isopentane, 1-Propanol, Pentane and Ethanol. 

Each solvent is put in a separate bottle. The lipid coated slides are prepared as previously described. For each solvent, a slide is put in the corresponding bottle. After \SI{15}{\minute}, the slides are removed and examined. The results are documented in a list. When all streaks caused by the lipid layer disappeared, the solvent is tested in the next experiment.

\subsubsection{at cryogenic temperature}
\label{chapter:meltingtemp}


Solubility is temperature dependent. Most solutions are endothermic. this means energy is needed to solve another substance. Also the saturation point of the solution can change. Therefore, solubility needs to be tested at the same temperature as in application

The experiment is conducted at \SI{-140}{\degreeCelsius}. The solvents are given in liquid nitrogen cooled baths, which are regulated to the desired temperature. A slide is given into the cold solvent for \SI{15}{\minute}. Then the slide is examined for leftover streaks as before.

The freezing point of tested solvents are not all below \SI{140}{\degreeCelsius}] (Table \ref{table:SchmelztemperaturLösungsmittel}). Still, solvents with a high freezing point can be mixed with other solvents with lower freezing point to lower the freezing point of the mixture. Alternatively, the temperature can be raised over the freezing point, but this could risk the ice to loose the vitrified state.

\begin{table}[hbt!]
	\centering
	\begin{tabular}{|l|c|}
		\hline
		solvent & melting point in °C \\
		\hline
		\hline
		4-Methyl Pentene & -154 \\ 
		\hline
		3-Methyl Pentene & -154 \\
		\hline
		1-Pentene & -165 \\
		\hline
		Isopentane & -160 \\
		\hline
		1-Propanol & -126 \\
		\hline
		Pentane & -129 \\
		\hline
		Ethanol & -114 \\
		\hline
	\end{tabular}
	\caption{Melting Point in °C for tested solvents.}
	\label{table:SchmelztemperaturLösungsmittel}
\end{table}

In the end, experiments has proven that solving lipids fast and reliable is not possible with tested solvents. As all solvent lipid combinations are endothermic. finding a working solvent lipid combination is very unlikely, as almost all combination of solvent and lipids will be endothermic.

Additionally, some solvents tested are soluble in water. It is unknown whether the solvents could be solved or diffuse inside the ice layer at \SI{-140}{\degreeCelsius}. Therefore the ice layer could be changed in some undesired manner. if a sufficient solvent is found and the solvent is soluble in water, a potential change of the vitrified ice needs to be addressed.
