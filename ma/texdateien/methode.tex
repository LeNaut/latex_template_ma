Das ist die Methode.

\section{Task and requirements}

\section{Lipids}

\subsection{Preparation lipids and cover glass}

\subsection{solubility lipids}

\subsubsection{at room temperature}

\subsubsection{at cryogenic temperature}

\section{"finger"}

\subsection{Plunge-freezing???}

\subsection{determining needed amount of glue}

First, the HFE used as glue was applied with a pincer. TO archieve this, the HFE is given in a cold bath at -140°C. This stops HFE from evaporating. The thickened hfe is now scooped with pincers on the tip of the finger. Though the correct amount is only determinable qualitative.

As an effort to determine the correct amount of HFE a pipette was used. ...


\subsection{temperature test}

\subsection{detaching ice from lipids}

\section{PDMS}

Why PDMS was chosen

\subsection{Preparation of PDMS samples}

All PDMS samples with different mixtures are prepared in the same way. The preparation starts with weighting out the needed amount of base coat and curing agent. The mixture is now stirred intensively. Then the mixture is placed in a vacuum bell for 30 minutes to gas out air bubbles. Meanwhile the cover glasses used as slides are cleaned with ethanol or isopropanol. Afterwards, the DPMS mixture is coat-spinned on the cover glass for 5 seconds with 300 rpm and then 120 seconds with 3000 rpm. Then the coated cover glasses are baked in the oven for 30 minutes exept a mixture of 1 base coat and 2 curing agent mixing ratio PDMS. For those 24 hours are needed, as it takes longer to harden and it will result in unwanted effects at plasma curing.

\subsection{Influence of plasma treatment on PDMS}

\subsubsection{setup}

\subsection{detaching ice from PDMS}

1:2 and 4:1

