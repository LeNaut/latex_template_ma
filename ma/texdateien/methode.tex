das ist die methode.

\subsection{klebermengentest}
Am Anfang wurde die Klebermenge mithilfe einer Pinzette dosiert. Hierfür wurde das HFE zuallererst in ein Bad gegeben, welches auf -140??°C gekühlt wurde. Bei diesen Temperaturen verdampft das HFE nicht mehr. außerdem wird das HFE Dickflüssig. Mit einer Pinzette wird dann nun die benötigte Menge auf die Fingerspitze aufgetragen. Jedoch kann die genaue Menge auf dem Finger nur qualitativ abgeschätzt werden.

Um die optimale Klebermenge quantitativ zu bestimmen, wurde das HFE mit Hilfe einer Pipette (MARKE) Pipettiert. 

\subsection{PDMS Coatspinnen}

ür eine 4:1 Beschichten eines Glases wird folgendes
Verfahren angewendet: zwächt wird das PDMS Fr
verfalls Thier 4 base oral eat curing agent gentles
verhält mir gewogen und gut zusammengemischt.
zum entfernen der Luftblasen wird das PDMS für
38 minuten in eine Vakuumglode gegeben. Die
der beschichtenen Glasbielfräser werden er lange mit
ethanol/1: Propurl gereinigt. Das ODM)wird
anschliebend mit eine Zoatspinner auf die Glaubjekt-
träger dünn aufgetragen. Anschlistend wird das PDNS
b 38 mir bei 880)ausgehärkt.
Für ein Mischungsverhältnis war 1:2 wird eine deutlich
Längere Aushärtzeit von mindöt 4h. benötigt Sand hält das
Das will lampelt an, weckes unter anderem bei der Planer
alhiering in unbeabsichtigte effeffle resultie